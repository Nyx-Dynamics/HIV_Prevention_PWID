 @article{Alpren_Dawson_John_Cranston_Panneer_Fukuda_Roosevelt_Klevens_Bryant_Peters_et al._2020, title={Opioid Use Fueling HIV Transmission in an Urban Setting: An Outbreak of HIV Infection Among People Who Inject Drugs—Massachusetts, 2015–2018}, volume={110}, ISSN={0090-0036, 1541-0048}, DOI={10.2105/AJPH.2019.305366}, abstractNote={Objectives. To describe and control an outbreak of HIV infection among people who inject drugs (PWID).
            Methods. The investigation included people diagnosed with HIV infection during 2015 to 2018 linked to 2 cities in northeastern Massachusetts epidemiologically or through molecular analysis. Field activities included qualitative interviews regarding service availability and HIV risk behaviors.
            Results. We identified 129 people meeting the case definition; 116 (90%) reported injection drug use. Molecular surveillance added 36 cases to the outbreak not otherwise linked. The 2 largest molecular groups contained 56 and 23 cases. Most interviewed PWID were homeless. Control measures, including enhanced field epidemiology, syringe services programming, and community outreach, resulted in a significant decline in new HIV diagnoses.
            Conclusions. We illustrate difficulties with identification and characterization of an outbreak of HIV infection among a population of PWID and the value of an intensive response.
            Public Health Implications. Responding to and preventing outbreaks requires ongoing surveillance, with timely detection of increases in HIV diagnoses, community partnerships, and coordinated services, all critical to achieving the goal of the national Ending the HIV Epidemic initiative.}, number={1}, journal={American Journal of Public Health}, author={Alpren, Charles and Dawson, Erica L. and John, Betsey and Cranston, Kevin and Panneer, Nivedha and Fukuda, H. Dawn and Roosevelt, Kathleen and Klevens, R. Monina and Bryant, Janice and Peters, Philip J. and Lyss, Sheryl B. and Switzer, William M. and Burrage, Amanda and Murray, Ashley and Agnew-Brune, Christine and Stiles, Tracy and McClung, Paul and Campbell, Ellsworth M. and Breen, Courtney and Randall, Liisa M. and Dasgupta, Sharoda and Onofrey, Shauna and Bixler, Danae and Hampton, Kischa and Jaeger, Jenifer Leaf and Hsu, Katherine K. and Adih, William and Callis, Barry and Goldman, Linda R. and Danner, Susie P. and Jia, Hongwei and Tumpney, Matthew and Board, Amy and Brown, Catherine and DeMaria, Alfred and Buchacz, Kate}, year={2020}, month=jan, pages={37–44}, language={en} }
 @article{Altice_Azbel_Stone_Brooks-Pollock_Smyrnov_Dvoriak_Taxman_El-Bassel_Martin_Booth_et al._2016, title={The perfect storm: incarceration and the high-risk environment perpetuating transmission of HIV, hepatitis C virus, and tuberculosis in Eastern Europe and Central Asia}, volume={388}, ISSN={01406736}, DOI={10.1016/S0140-6736(16)30856-X}, number={10050}, journal={The Lancet}, author={Altice, Frederick L and Azbel, Lyuba and Stone, Jack and Brooks-Pollock, Ellen and Smyrnov, Pavlo and Dvoriak, Sergii and Taxman, Faye S and El-Bassel, Nabila and Martin, Natasha K and Booth, Robert and Stöver, Heino and Dolan, Kate and Vickerman, Peter}, year={2016}, month=sept, pages={1228–1248}, language={en} }
 @article{Ances_Ortega_Vaida_Heaps_Paul_2012, title={Independent Effects of HIV, Aging, and HAART on Brain Volumetric Measures}, volume={59}, ISSN={1525-4135}, DOI={10.1097/QAI.0b013e318249db17}, number={5}, journal={JAIDS Journal of Acquired Immune Deficiency Syndromes}, author={Ances, Beau M. and Ortega, Mario and Vaida, Florin and Heaps, Jodi and Paul, Robert}, year={2012}, month=apr, pages={469–477}, language={en} }
 @article{Anderson_Harezlak_Bharti_Mi_Taylor_Daar_Schifitto_Zhong_Alger_Brown_et al._2015, title={Plasma and Cerebrospinal Fluid Biomarkers Predict Cerebral Injury in HIV-Infected Individuals on Stable Combination Antiretroviral Therapy}, volume={69}, ISSN={1525-4135}, DOI={10.1097/QAI.0000000000000532}, number={1}, journal={JAIDS Journal of Acquired Immune Deficiency Syndromes}, author={Anderson, Albert M. and Harezlak, Jaroslaw and Bharti, Ajay and Mi, Deming and Taylor, Michael J. and Daar, Eric S. and Schifitto, Giovanni and Zhong, Jianhui and Alger, Jeffry R. and Brown, Mark S. and Singer, Elyse J. and Campbell, Thomas B. and McMahon, Deborah D. and Buchthal, Steven and Cohen, Ronald and Yiannoutsos, Constantin and Letendre, Scott L. and Navia, Bradford A.}, year={2015}, month=may, pages={29–35}, language={en} }
 @article{Arum_Fraser_Artenie_Bivegete_Trickey_Alary_Astemborski_Iversen_Lim_MacGregor_et al._2021, title={Homelessness, unstable housing, and risk of HIV and hepatitis C virus acquisition among people who inject drugs: a systematic review and meta-analysis}, volume={6}, ISSN={24682667}, DOI={10.1016/S2468-2667(21)00013-X}, number={5}, journal={The Lancet Public Health}, author={Arum, Chiedozie and Fraser, Hannah and Artenie, Andreea Adelina and Bivegete, Sandra and Trickey, Adam and Alary, Michel and Astemborski, Jacquie and Iversen, Jennifer and Lim, Aaron G and MacGregor, Louis and Morris, Meghan and Ong, Jason J and Platt, Lucy and Sack-Davis, Rachel and Van Santen, Daniela K and Solomon, Sunil S and Sypsa, Vana and Valencia, Jorge and Van Den Boom, Wijnand and Walker, Josephine G and Ward, Zoe and Stone, Jack and Vickerman, Peter and Cherutich, Peter and Debeck, Kora and Dietze, Paul and Dumchev, Kostyantyn and Hayashi, Kanna and Hellard, Margaret and Hickman, Matthew and Hope, Vivian and Judd, Ali and Kåberg, Martin and Kurth, Ann E. and Leclerc, Pascale and Maher, Lisa and Mehta, Shruti H. and Page, Kimberly A and Prins, Maria and Todd, Catherine S. and Strathdee, Steffanie A.}, year={2021}, month=may, pages={e309–e323}, language={en} }
 @article{Baugher_Wejnert_Kanny_Broz_Feelemyer_Hershow_Burnett_Chapin-Bardales_Haynes_Finlayson_et al._2025, title={Are we ending the HIV epidemic among persons who inject drugs?: key findings from 19 US cities}, volume={39}, ISSN={0269-9370, 1473-5571}, DOI={10.1097/QAD.0000000000004249}, abstractNote={Objectives:
              National HIV Behavioral Surveillance (NHBS) conducts surveillance among key populations, including persons who inject drugs (PWID). NHBS data can be used to monitor progress toward national goals, including Ending the HIV Epidemic (EHE). EHE strategies include HIV testing (Diagnose), rapid linkage to HIV treatment (Treat), and increasing access to preexposure prophylaxis (PrEP), and syringe services programs (SSPs) (Prevent). This analysis aimed to concisely compare NHBS key findings among PWID to EHE goals.
            
            
              Design/methods:
              
                Cross-sectional NHBS data were collected from PWID in 2018 (
                n
                 = 9786) and 2022 (
                n
                 = 6574) in 19 US cities. We compared key findings from 2022 NHBS to specified EHE goals for Diagnose (HIV testing) and Treat (linkage to care, current antiretroviral therapy (ART) use) or 2018 NHBS key findings for Prevent (PrEP and SSP use).
              
            
            
              Results:
              In 2022, 45% of PWID were tested for HIV; 45% of PWID with HIV were linked to care within 1 month of diagnosis, and 79% were currently taking ART; 1% of PWID without HIV used PrEP; and approximately half of all PWID received syringes from an SSP. PrEP and SSP use among PWID have not changed since 2018.
            
            
              Conclusion:
              National HIV strategies are not yet adequately reaching PWID. To end the US HIV epidemic, multilevel solutions are needed to tailor interventions for PWID and dismantle barriers to testing, treatment, and prevention. Structural solutions to improve access to basic needs and SSPs may have downstream benefits across the EHE strategies.}, number={12}, journal={AIDS}, author={Baugher, Amy R. and Wejnert, Cyprian and Kanny, Dafna and Broz, Dita and Feelemyer, Jonathan and Hershow, Rebecca B. and Burnett, Janet and Chapin-Bardales, Johanna and Haynes, Maya and Finlayson, Teresa and Prejean, Joseph and for the NHBS Study Group*}, year={2025}, month=oct, pages={1813–1819}, language={en} }
 @article{Bazzi_Biancarelli_Childs_Drainoni_Edeza_Salhaney_Mimiaga_Biello_2018, title={Limited Knowledge and Mixed Interest in Pre-Exposure Prophylaxis for HIV Prevention Among People Who Inject Drugs}, volume={32}, rights={https://www.liebertpub.com/nv/resources-tools/text-and-data-mining-policy/121/}, ISSN={1087-2914, 1557-7449}, DOI={10.1089/apc.2018.0126}, number={12}, journal={AIDS Patient Care and STDs}, author={Bazzi, Angela R. and Biancarelli, Dea L. and Childs, Ellen and Drainoni, Mari-Lynn and Edeza, Alberto and Salhaney, Peter and Mimiaga, Matthew J. and Biello, Katie B.}, year={2018}, month=dec, pages={529–537}, language={en} }
 @article{Bazzi_Drainoni_Biancarelli_Hartman_Mimiaga_Mayer_Biello_2019, title={Systematic review of HIV treatment adherence research among people who inject drugs in the United States and Canada: evidence to inform pre-exposure prophylaxis (PrEP) adherence interventions}, volume={19}, ISSN={1471-2458}, DOI={10.1186/s12889-018-6314-8}, number={1}, journal={BMC Public Health}, author={Bazzi, Angela R. and Drainoni, Mari-Lynn and Biancarelli, Dea L. and Hartman, Joshua J. and Mimiaga, Matthew J. and Mayer, Kenneth H. and Biello, Katie B.}, year={2019}, month=dec, pages={31}, language={en} }
 @article{Bazzi_Shaw_Biello_Vahey_Brody_2023, title={Patient and Provider Perspectives on a Novel, Low-Threshold HIV PrEP Program for People Who Inject Drugs Experiencing Homelessness}, volume={38}, ISSN={0884-8734, 1525-1497}, DOI={10.1007/s11606-022-07672-5}, abstractNote={Abstract
            
              Background
              HIV outbreaks among people who inject drugs (PWID) and experience homelessness are increasing across the USA. Despite high levels of need, multilevel barriers to accessing antiretroviral pre-exposure prophylaxis (PrEP) for HIV prevention persist for this population. The Boston Health Care for the Homeless Program (BHCHP) initiated a low-threshold, outreach-based program to support engagement in PrEP services among PWID experiencing homelessness.
            
            
              Methods
              To inform dissemination efforts, we explored patient and provider perspectives on key program components. From March to December 2020, we conducted semi-structured qualitative interviews with current and former BHCHP PrEP program participants and prescribers, patient navigators, and outreach workers (i.e., providers). Thematic analysis explored perspectives on key program components.
            
            
              Results
              
                Participants (
                n
                = 21) and providers (
                n
                = 11) identified the following five key components of BHCHP’s PrEP program that they perceived to be particularly helpful for supporting patient engagement in PrEP services: (1) community-driven PrEP education; (2) low-threshold, accessible programming including same-day PrEP prescribing; (3) tailored prescribing supports (e.g., on-site pharmacy, short-term prescriptions, medication storage); (4) intensive outreach and navigation; and (5) trusting, respectful patient-provider relationships.
              
            
            
              Discussion
              Findings suggest that more patient-centered services formed the basis of BHCHP’s innovative, successful PrEP program. While contextual challenges including competing public health emergencies and homeless encampment “sweeps” necessitate ongoing programmatic adaptations, lessons from BHCHP’s PrEP program can inform PrEP delivery in a range of community-based settings serving this population, including syringe service programs and shelters.}, number={4}, journal={Journal of General Internal Medicine}, author={Bazzi, Angela R. and Shaw, Leah C. and Biello, Katie B. and Vahey, Seamus and Brody, Jennifer K.}, year={2023}, month=mar, pages={913–921}, language={en} }
 @article{Bekker_Das_Abdool Karim_Ahmed_Batting_Brumskine_Gill_Harkoo_Jaggernath_Kigozi_et al._2024, title={Twice-Yearly Lenacapavir or Daily F/TAF for HIV Prevention in Cisgender Women}, volume={391}, rights={http://www.nejmgroup.org/legal/terms-of-use.htm}, ISSN={0028-4793, 1533-4406}, DOI={10.1056/NEJMoa2407001}, number={13}, journal={New England Journal of Medicine}, author={Bekker, Linda-Gail and Das, Moupali and Abdool Karim, Quarraisha and Ahmed, Khatija and Batting, Joanne and Brumskine, William and Gill, Katherine and Harkoo, Ishana and Jaggernath, Manjeetha and Kigozi, Godfrey and Kiwanuka, Noah and Kotze, Philip and Lebina, Limakatso and Louw, Cheryl E. and Malahleha, Moelo and Manentsa, Mmatsie and Mansoor, Leila E. and Moodley, Dhayendre and Naicker, Vimla and Naidoo, Logashvari and Naidoo, Megeshinee and Nair, Gonasagrie and Ndlovu, Nkosiphile and Palanee-Phillips, Thesla and Panchia, Ravindre and Pillay, Saresha and Potloane, Disebo and Selepe, Pearl and Singh, Nishanta and Singh, Yashna and Spooner, Elizabeth and Ward, Amy M. and Zwane, Zwelethu and Ebrahimi, Ramin and Zhao, Yang and Kintu, Alexander and Deaton, Chris and Carter, Christoph C. and Baeten, Jared M. and Matovu Kiweewa, Flavia}, year={2024}, month=oct, pages={1179–1192}, language={en} }
 @article{Biancarelli_Biello_Childs_Drainoni_Salhaney_Edeza_Mimiaga_Saitz_Bazzi_2019, title={Strategies used by people who inject drugs to avoid stigma in healthcare settings}, volume={198}, ISSN={03768716}, DOI={10.1016/j.drugalcdep.2019.01.037}, journal={Drug and Alcohol Dependence}, author={Biancarelli, Dea L. and Biello, Katie B. and Childs, Ellen and Drainoni, M. and Salhaney, Peter and Edeza, Alberto and Mimiaga, Matthew J and Saitz, Richard and Bazzi, Angela R.}, year={2019}, month=may, pages={80–86}, language={en} }
 @article{Biello_Bazzi_Mimiaga_Biancarelli_Edeza_Salhaney_Childs_Drainoni_2018, title={Perspectives on HIV pre-exposure prophylaxis (PrEP) utilization and related intervention needs among people who inject drugs}, volume={15}, ISSN={1477-7517}, DOI={10.1186/s12954-018-0263-5}, abstractNote={BACKGROUND: Antiretroviral pre-exposure prophylaxis (PrEP) is clinically efficacious and recommended for HIV prevention among people who inject drugs (PWID), but uptake remains low and intervention needs are understudied. To inform the development of PrEP interventions for PWID, we conducted a qualitative study in the Northeastern USA, a region where recent clusters of new HIV infections have been attributed to injection drug use.
METHODS: We conducted qualitative interviews with 33 HIV-uninfected PWID (hereafter, “participants”) and 12 clinical and social service providers (professional “key informants”) in Boston, MA, and Providence, RI, in 2017. Trained interviewers used semi-structured interviews to explore PrEP acceptability and perceived barriers to use. Thematic analysis of coded data identified multilevel barriers to PrEP use among PWID and related intervention strategies.
RESULTS: Among PWID participants (n = 33, 55% male), interest in PrEP was high, but both participants and professional key informants (n = 12) described barriers to PrEP utilization that occurred at one or more socioecological levels. Individual-level barriers included low PrEP knowledge and limited HIV risk perception, concerns about PrEP side effects, and competing health priorities and needs due to drug use and dependence. Interpersonal-level barriers included negative experiences with healthcare providers and HIV-related stigma within social networks. Clinical barriers included poor infrastructure and capacity for PrEP delivery to PWID, and structural barriers related to homelessness, criminal justice system involvement, and lack of money or identification to get prescriptions. Participants and key informants provided some suggestions for strategies to address these multilevel barriers and better facilitate PrEP delivery to PWID.
CONCLUSIONS: In addition to some of the facilitators of PrEP use identified by participants and key informants, we drew on our key findings and behavioral change theory to propose additional intervention targets. In particular, to help address the multilevel barriers to PrEP uptake and adherence, we discuss ways that interventions could target information, self-regulation and self-efficacy, social support, and environmental change. PrEP is clinically efficacious and has been recommended for PWID; thus, development and testing of strategies to improve PrEP delivery to this high-risk and socially marginalized population are needed.}, number={1}, journal={Harm Reduction Journal}, author={Biello, K. B. and Bazzi, A. R. and Mimiaga, M. J. and Biancarelli, D. L. and Edeza, A. and Salhaney, P. and Childs, E. and Drainoni, M. L.}, year={2018}, month=nov, pages={55}, language={eng} }
 @article{Biello_Edeza_Salhaney_Biancarelli_Mimiaga_Drainoni_Childs_Bazzi_2019, title={A missing perspective: injectable pre-exposure prophylaxis for people who inject drugs}, volume={31}, ISSN={0954-0121, 1360-0451}, DOI={10.1080/09540121.2019.1587356}, number={10}, journal={AIDS Care}, author={Biello, K. B. and Edeza, A. and Salhaney, P. and Biancarelli, D. L. and Mimiaga, M. J. and Drainoni, M. L. and Childs, E. S. and Bazzi, A. R.}, year={2019}, month=oct, pages={1214–1220}, language={en} }
 @article{Bonacci_Moorman_Bixler_Penley_Wilson_Hudson_McClung_2023, title={Prevention and Care Opportunities for People Who Inject Drugs in an HIV Outbreak - Kanawha County, West Virginia, 2019-2021}, volume={38}, ISSN={1525-1497}, DOI={10.1007/s11606-022-07875-w}, number={3}, journal={Journal of General Internal Medicine}, author={Bonacci, Robert A. and Moorman, Anne C. and Bixler, Danae and Penley, McKenna and Wilson, Suzanne and Hudson, Alana and McClung, R. Paul}, year={2023}, month=feb, pages={828–831}, language={eng} }
 @article{Brody_Taylor_Biello_Bazzi_2021, title={Towards equity for people who inject drugs in HIV prevention drug trials}, volume={96}, ISSN={09553959}, DOI={10.1016/j.drugpo.2021.103284}, journal={International Journal of Drug Policy}, author={Brody, Jennifer Karen and Taylor, Jessica and Biello, Katie and Bazzi, Angela R.}, year={2021}, month=oct, pages={103284}, language={en} }
 @misc{Centers for Disease Control and Prevention_2021, title={US Public Health Service: Preexposure Prophylaxis for the Prevention of HIV Infection in the United States—2021 Update: A Clinical Practice Guideline}, url={https://www.cdc.gov/hiv/pdf/risk/prep/cdc-hiv-prep-guidelines-2021.pdf}, author={Centers for Disease Control and Prevention}, year={2021} }
 @article{Choopanya_Martin_Suntharasamai_Sangkum_Mock_Leethochawalit_Chiamwongpaet_Kitisin_Natrujirote_Kittimunkong_et al._2013, title={Antiretroviral prophylaxis for HIV infection in injecting drug users in Bangkok, Thailand (the Bangkok Tenofovir Study): a randomised, double-blind, placebo-controlled phase 3 trial}, volume={381}, ISSN={01406736}, DOI={10.1016/S0140-6736(13)61127-7}, number={9883}, journal={The Lancet}, author={Choopanya, Kachit and Martin, Michael and Suntharasamai, Pravan and Sangkum, Udomsak and Mock, Philip A and Leethochawalit, Manoj and Chiamwongpaet, Sithisat and Kitisin, Praphan and Natrujirote, Pitinan and Kittimunkong, Somyot and Chuachoowong, Rutt and Gvetadze, Roman J and McNicholl, Janet M and Paxton, Lynn A and Curlin, Marcel E and Hendrix, Craig W and Vanichseni, Suphak}, year={2013}, month=june, pages={2083–2090}, language={en} }
 @article{Cranston_Alpren_John_Dawson_Roosevelt_Burrage_Bryant_Switzer_Breen_Peters_et al._2019, title={Notes from the Field: HIV Diagnoses Among Persons Who Inject Drugs — Northeastern Massachusetts, 2015–2018}, volume={68}, ISSN={0149-2195, 1545-861X}, DOI={10.15585/mmwr.mm6810a6}, number={10}, journal={MMWR. Morbidity and Mortality Weekly Report}, author={Cranston, Kevin and Alpren, Charles and John, Betsey and Dawson, Erica and Roosevelt, Kathleen and Burrage, Amanda and Bryant, Janice and Switzer, William M. and Breen, Courtney and Peters, Philip J. and Stiles, Tracy and Murray, Ashley and Fukuda, H. Dawn and Adih, William and Goldman, Linda and Panneer, Nivedha and Callis, Barry and Campbell, Ellsworth M. and Randall, Liisa and France, Anne Marie and Klevens, R. Monina and Lyss, Sheryl and Onofrey, Shauna and Agnew-Brune, Christine and Goulart, Michael and Jia, Hongwei and Tumpney, Matthew and McClung, Paul and Dasgupta, Sharoda and Bixler, Danae and Hampton, Kischa and Amy Board and Jaeger, Jenifer Leaf and Buchacz, Kate and DeMaria, Alfred}, year={2019}, month=mar, pages={253–254} }
 @article{DeBeck_Cheng_Montaner_Beyrer_Elliott_Sherman_Wood_Baral_2017, title={HIV and the criminalisation of drug use among people who inject drugs: a systematic review}, volume={4}, ISSN={23523018}, DOI={10.1016/S2352-3018(17)30073-5}, number={8}, journal={The Lancet HIV}, author={DeBeck, Kora and Cheng, Tessa and Montaner, Julio S and Beyrer, Chris and Elliott, Richard and Sherman, Susan and Wood, Evan and Baral, Stefan}, year={2017}, month=aug, pages={e357–e374}, language={en} }
 @article{Degenhardt_Peacock_Colledge_Leung_Grebely_Vickerman_Stone_Cunningham_Trickey_Dumchev_et al._2017, title={Global prevalence of injecting drug use and sociodemographic characteristics and prevalence of HIV, HBV, and HCV in people who inject drugs: a multistage systematic review}, volume={5}, ISSN={2214109X}, DOI={10.1016/S2214-109X(17)30375-3}, number={12}, journal={The Lancet Global Health}, author={Degenhardt, Louisa and Peacock, Amy and Colledge, Samantha and Leung, Janni and Grebely, Jason and Vickerman, Peter and Stone, Jack and Cunningham, Evan B and Trickey, Adam and Dumchev, Kostyantyn and Lynskey, Michael and Griffiths, Paul and Mattick, Richard P and Hickman, Matthew and Larney, Sarah}, year={2017}, month=dec, pages={e1192–e1207}, language={en} }
 @article{Delany-Moretlwe_Hughes_Bock_Ouma_Hunidzarira_Kalonji_Kayange_Makhema_Mandima_Mathebula_et al._2022, title={Cabotegravir for the prevention of HIV-1 in women: Results from HPTN 084}, volume={387}, DOI={10.1056/NEJMoa2115829}, journal={N. Engl. J. Med.}, author={Delany-Moretlwe, S. and Hughes, J. P. and Bock, P. and Ouma, S. G. and Hunidzarira, P. and Kalonji, D. and Kayange, N. and Makhema, J. and Mandima, P. and Mathebula, M. and others}, year={2022}, pages={2043–2055} }
 @article{Des Jarlais_Feelemyer_LaKosky_Szymanowski_Arasteh_2020, title={Expansion of syringe service programs in the United States, 2015–2018}, volume={110}, DOI={10.2105/AJPH.2019.305515}, journal={Am. J. Public Health}, author={Des Jarlais, D. C. and Feelemyer, J. and LaKosky, P. and Szymanowski, K. and Arasteh, K.}, year={2020}, pages={517–519} }
 @article{Dore_Altice_Litwin_Dalgard_Gane_Shibolet_Luetkemeyer_Nahass_Peng_Conway_et al._2016, title={Elbasvir–Grazoprevir to Treat Hepatitis C Virus Infection in Persons Receiving Opioid Agonist Therapy: A Randomized Trial}, volume={165}, ISSN={0003-4819}, DOI={10.7326/M16-0816}, number={9}, journal={Annals of Internal Medicine}, author={Dore, Gregory J. and Altice, Frederick and Litwin, Alain H. and Dalgard, Olav and Gane, Edward J. and Shibolet, Oren and Luetkemeyer, Anne and Nahass, Ronald and Peng, Cheng-Yuan and Conway, Brian and Grebely, Jason and Howe, Anita Y.M. and Gendrano, Isaias N. and Chen, Erluo and Huang, Hsueh-Cheng and Dutko, Frank J. and Nickle, David C. and Nguyen, Bach-Yen and Wahl, Janice and Barr, Eliav and Robertson, Michael N. and Platt, Heather L. and on behalf of the C-EDGE CO-STAR Study Group}, year={2016}, month=nov, pages={625}, language={en} }
 @article{Eger_Bazzi_Valasek_Vera_Harvey-Vera_Artamonova_Rangel_Strathdee_Pines_2024, title={Long-acting Injectable PrEP Interest and General PrEP Awareness among People who Inject Drugs in the San Diego-Tijuana Border Metroplex}, volume={28}, ISSN={1573-3254}, DOI={10.1007/s10461-024-04285-3}, abstractNote={Long-acting injectable HIV pre-exposure prophylaxis (LAI-PrEP) could help overcome multilevel challenges to HIV prevention for people who inject drugs (PWID), including those in the binational San Diego-Tijuana metroplex. Yet, general PrEP awareness and interest in LAI-PrEP remain underexplored among PWID. From 2020 to 2021, 562 HIV-negative PWID in San Diego and Tijuana completed surveys assessing general PrEP awareness and interest in oral and LAI-PrEP. Modified Poisson regression examined factors associated with general PrEP awareness. Multinomial logistic regression assessed factors associated with interest in both oral and LAI-PrEP, oral PrEP only, LAI-PrEP only, or neither. General PrEP awareness was low (18%) and associated with experiencing unsheltered homelessness (adjusted prevalence ratio [APR] = 1.50, 95% confidence interval [CI]: 0.96-2.33), past 6-month fentanyl injection (APR = 1.53, 95% CI: 1.04-2.25), and transactional sex (APR = 1.71, 95% CI: 1.06-2.76). Interest in oral PrEP only was most common (44%), followed by LAI-PrEP only (25%) and neither (16%). Compared to the odds of being interested in LAI-PrEP only, the odds of being interested in oral PrEP only were lower among those who were stopped by police (AOR = 0.38, 95% CI: 0.22-0.65), reported past 6-month fentanyl injection (AOR = 0.33, 95% CI: 0.20-0.56), polydrug use (AOR = 0.48, 95% CI: 0.27-0.86), injecting multiple times daily (AOR = 0.26, 95% CI: 0.14-0.46), receptive syringe use (AOR = 0.30, 95% CI: 0.19-0.49), and higher perceived HIV risk (AOR = 0.24, 95% CI: 0.15-0.39). Interest in LAI-PrEP was more common among PWID reporting social and structural factors that could interfere with oral PrEP adherence, suggesting LAI-PrEP implementation could increase PrEP coverage among those most vulnerable to HIV.}, number={5}, journal={AIDS and behavior}, author={Eger, William H. and Bazzi, Angela R. and Valasek, Chad J. and Vera, Carlos F. and Harvey-Vera, Alicia and Artamonova, Irina and Rangel, M. Gudelia and Strathdee, Steffanie A. and Pines, Heather A.}, year={2024}, month=may, pages={1650–1661}, language={eng} }
 @article{Eshleman_Fogel_Halvas_Piwowar-Manning_Marzinke_Kofron_Wang_Mellors_McCauley_Rinehart_et al._2022, title={HIV RNA Screening Reduces Integrase Strand Transfer Inhibitor Resistance Risk in Persons Receiving Long-Acting Cabotegravir for HIV Prevention}, volume={226}, rights={https://academic.oup.com/pages/standard-publication-reuse-rights}, ISSN={0022-1899, 1537-6613}, DOI={10.1093/infdis/jiac415}, abstractNote={Abstract
            
              Background
              The HPTN 083 trial demonstrated that long-acting cabotegravir (CAB-LA) was superior to tenofovir-disoproxil fumarate/emtricitabine for human immunodeficiency virus (HIV) preexposure prophylaxis (PrEP). Integrase strand transfer inhibitor (INSTI) resistance-associated mutations (RAMs) were detected in some participants with HIV infection. We used a low viral load INSTI genotyping assay to evaluate the timing of emergence of INSTI RAMs and assessed whether HIV screening with a sensitive RNA assay would have detected HIV infection before INSTI resistance emerged.
            
            
              Methods
              Single-genome sequencing to detect INSTI RAMs was performed for samples with viral loads &lt;500 copies/mL from 5 participants with previously identified INSTI RAMs and 2 with no prior genotyping results.
            
            
              Results
              Major INSTI RAMs were detected in all 7 cases. HIV RNA testing identified infection before major INSTI RAMs emerged in 4 cases and before additional major INSTI RAMs accumulated in 1 case. Most INSTI RAMs were detected early when the viral load was low and CAB concentration was high.
            
            
              Conclusions
              When using CAB-LA PrEP, earlier detection of HIV infection with a sensitive RNA assay may allow for earlier treatment initiation with the potential to reduce INSTI resistance risk. Further studies are needed to evaluate the value and feasibility of HIV RNA testing with CAB-LA PrEP.}, number={12}, journal={The Journal of Infectious Diseases}, author={Eshleman, Susan H and Fogel, Jessica M and Halvas, Elias K and Piwowar-Manning, Estelle and Marzinke, Mark A and Kofron, Ryan and Wang, Zhe and Mellors, John and McCauley, Marybeth and Rinehart, Alex R and St Clair, Marty and Adeyeye, Adeola and Hinojosa, Juan C and Cabello, Robinson and Middelkoop, Keren and Hanscom, Brett and Cohen, Myron S and Grinsztejn, Beatriz and Landovitz, Raphael J and HPTN 083 Study Team and Seisa, Michael and Lie, Yolanda and Meyer, William and Marrazzo, Jeanne and Peel, Sheila and Wallis, Carole and Asmelash, Aida and Daar, Eric and Rooney, James and Clark, Richard}, year={2022}, month=dec, pages={2170–2180}, language={en} }
 @article{Glick_Burt_Kummer_Tinsley_Banta-Green_Golden_2018, title={Increasing methamphetamine injection among non-MSM who inject drugs in King County, Washington}, volume={182}, ISSN={03768716}, DOI={10.1016/j.drugalcdep.2017.10.011}, journal={Drug and Alcohol Dependence}, author={Glick, Sara Nelson and Burt, Richard and Kummer, Kim and Tinsley, Joe and Banta-Green, Caleb J. and Golden, Matthew R.}, year={2018}, month=jan, pages={86–92}, language={en} }
 @article{Golden_Lechtenberg_Glick_Dombrowski_Duchin_Reuer_Dhanireddy_Neme_Buskin_2019, title={Outbreak of Human Immunodeficiency Virus Infection Among Heterosexual Persons Who Are Living Homeless and Inject Drugs — Seattle, Washington, 2018}, volume={68}, ISSN={0149-2195, 1545-861X}, DOI={10.15585/mmwr.mm6815a2}, number={15}, journal={MMWR. Morbidity and Mortality Weekly Report}, author={Golden, Matthew R. and Lechtenberg, Richard and Glick, Sara N. and Dombrowski, Julie and Duchin, Jeff and Reuer, Jennifer R. and Dhanireddy, Shireesha and Neme, Santiago and Buskin, Susan E.}, year={2019}, month=apr, pages={344–349} }
 @article{Gonsalves_Crawford_2018, title={Dynamics of the HIV outbreak and response in Scott County, IN, USA, 2011–15: a modelling study}, volume={5}, ISSN={23523018}, DOI={10.1016/S2352-3018(18)30176-0}, number={10}, journal={The Lancet HIV}, author={Gonsalves, Gregg S and Crawford, Forrest W}, year={2018}, month=oct, pages={e569–e577}, language={en} }
 @article{Grant_Atkinson_Hesselink_Kennedy_Richman_Spector_McCUTCHAN_1987, title={Evidence for Early Central Nervous System Involvement in the Acquired Immunodeficiency Syndrome (AIDS) and Other Human Immunodeficiency Virus (HIV) Infections: Studies with Neuropsychologic Testing and Magnetic Resonance Imaging}, volume={107}, rights={https://www.acpjournals.org/journal/aim/text-and-data-mining}, ISSN={0003-4819, 1539-3704}, DOI={10.7326/0003-4819-107-6-828}, number={6}, journal={Ann Intern Med}, author={Grant, Igor and Atkinson, J. Hampton and Hesselink, John R. and Kennedy, Caroline J. and Richman, Douglas D. and Spector, Stephen A. and McCUTCHAN, J. Allen}, year={1987}, month=dec, pages={828–836}, language={en} }
 @article{Grant_Lama_Anderson_McMahan_Liu_Vargas_Goicochea_Casapía_Guanira-Carranza_Ramirez-Cardich_et al._2010, title={Preexposure Chemoprophylaxis for HIV Prevention in Men Who Have Sex with Men}, volume={363}, ISSN={0028-4793, 1533-4406}, DOI={10.1056/NEJMoa1011205}, number={27}, journal={New England Journal of Medicine}, author={Grant, Robert M. and Lama, Javier R. and Anderson, Peter L. and McMahan, Vanessa and Liu, Albert Y. and Vargas, Lorena and Goicochea, Pedro and Casapía, Martín and Guanira-Carranza, Juan Vicente and Ramirez-Cardich, Maria E. and Montoya-Herrera, Orlando and Fernández, Telmo and Veloso, Valdilea G. and Buchbinder, Susan P. and Chariyalertsak, Suwat and Schechter, Mauro and Bekker, Linda-Gail and Mayer, Kenneth H. and Kallás, Esper Georges and Amico, K. Rivet and Mulligan, Kathleen and Bushman, Lane R. and Hance, Robert J. and Ganoza, Carmela and Defechereux, Patricia and Postle, Brian and Wang, Furong and McConnell, J. Jeff and Zheng, Jia-Hua and Lee, Jeanny and Rooney, James F. and Jaffe, Howard S. and Martinez, Ana I. and Burns, David N. and Glidden, David V.}, year={2010}, month=dec, pages={2587–2599}, language={en} }
 @article{Grebely_Dalgard_Conway_Cunningham_Bruggmann_Hajarizadeh_Amin_Bruneau_Hellard_Litwin_et al._2018, title={Sofosbuvir and velpatasvir for hepatitis C virus infection in people with recent injection drug use (SIMPLIFY): an open-label, single-arm, phase 4, multicentre trial}, volume={3}, ISSN={24681253}, DOI={10.1016/S2468-1253(17)30404-1}, number={3}, journal={The Lancet Gastroenterology & Hepatology}, author={Grebely, Jason and Dalgard, Olav and Conway, Brian and Cunningham, Evan B and Bruggmann, Philip and Hajarizadeh, Behzad and Amin, Janaki and Bruneau, Julie and Hellard, Margaret and Litwin, Alain H and Marks, Philippa and Quiene, Sophie and Siriragavan, Sharmila and Applegate, Tanya L and Swan, Tracy and Byrne, Jude and Lacalamita, Melanie and Dunlop, Adrian and Matthews, Gail V and Powis, Jeff and Shaw, David and Thurnheer, Maria Christine and Weltman, Martin and Kronborg, Ian and Cooper, Curtis and Feld, Jordan J and Fraser, Chris and Dillon, John F and Read, Phillip and Gane, Ed and Dore, Gregory J}, year={2018}, month=mar, pages={153–161}, language={en} }
 @article{Grov_Westmoreland_Morrison_Carrico_Nash_2020, title={The Crisis We Are Not Talking About: One-in-Three Annual HIV Seroconversions Among Sexual and Gender Minorities Were Persistent Methamphetamine Users}, volume={85}, ISSN={1525-4135}, DOI={10.1097/QAI.0000000000002461}, abstractNote={Introduction:
              Methamphetamine use is once again on the rise among sexual and gender minorities who have sex with men (SGMSM).
            
            
              Methods:
              Baseline and 12-month data are taken from an ongoing cohort study of n = 4786 SGMSM aged 16–49 at risk for HIV from across the United States. Participants completed annual online surveys and at-home HIV testing (oral fluid samples returned through mail).
            
            
              Results:
              Overall, 2.47 per 100 persons seroconverted over 12 months. In addition, 13.8% of participants reported any methamphetamine use over the 12-month study period. Nearly three-fourths (74.7%; 422 of 565) of those who reported using methamphetamine at baseline were persistent users at 12 months. In adjusted analyses, compared with those who did not use methamphetamine, incident methamphetamine users (ie, those who indicated use between baseline and follow-up) and persistent methamphetamine users had significantly higher odds of HIV seroconverting (adjusted odds ratio = 3.95, 95% confidence interval: 1.64 to 9.47; and 7.11, 4.53 to 11.17, respectively). Persistent methamphetamine users accounted for one-third of all observed HIV seroconversions (41 of 115).
            
            
              Discussion:
              Among SGMSM at elevated risk for HIV, persistent methamphetamine use was prevalent and associated with substantially amplified risk for HIV seroconversion. Expanded efforts are needed to test implementation strategies for scalable, evidence-based interventions to reduce HIV risk in SGMSM who use methamphetamine.}, number={3}, journal={JAIDS Journal of Acquired Immune Deficiency Syndromes}, author={Grov, Christian and Westmoreland, Drew and Morrison, Corey and Carrico, Adam W. and Nash, Denis}, year={2020}, month=nov, pages={272–279}, language={en} }
 @article{Hellmuth_Slike_Sacdalan_Best_Kroon_Phanuphak_Fletcher_Prueksakaew_Jagodzinski_Valcour_et al._2019, title={Very Early Initiation of Antiretroviral Therapy During Acute HIV Infection Is Associated With Normalized Levels of Immune Activation Markers in Cerebrospinal Fluid but Not in Plasma}, volume={220}, rights={https://academic.oup.com/journals/pages/open_access/funder_policies/chorus/standard_publication_model}, ISSN={0022-1899, 1537-6613}, DOI={10.1093/infdis/jiz030}, abstractNote={Abstract Background Chronic immune activation in the blood and central nervous system is a consequence of human immunodeficiency virus (HIV) infection that contributes to disease morbidity and can occur despite virally suppressive antiretroviral therapy (ART). The trajectory of HIV-related inflammation may vary with the timing of ART initiation. We examined immune activation markers in cerebrospinal fluid (CSF) and blood specimens collected over 96 weeks from participants who initiated ART during acute HIV infection (AHI). Methods RV254/SEARCH010 study participants with AHI underwent CSF (n = 89) and plasma (n = 146) sampling before initiating ART and at weeks 24 and 96 of treatment. A majority participants (64.4%) received a standard ART regimen (hereafter, “standard ART”), with some (34.7%) also receiving maraviroc and raltegravir for the first 24 weeks (hereafter, “ART plus”). We compared neopterin, CXCL10, CCL2, and interleukin 6 (IL-6) levels in the AHI group to those in 18 healthy, uninfected controls. Results Following 24 and 96 weeks of treatment, levels of all CSF markers normalized while levels of several plasma markers remained elevated in the AHI group (P textless .001). Participants receiving the ART-plus regimen had lower median plasma CCL2 levels at week 24 and lower plasma neopterin levels at week 96. Conclusions ART initiation during AHI differentially impacts the brain compartment, with markers of inflammation returning to normal levels in the CSF, where they were sustained at week 96, but not in plasma.}, number={12}, journal={The Journal of Infectious Diseases}, author={Hellmuth, Joanna and Slike, Bonnie M and Sacdalan, Carlo and Best, John and Kroon, Eugene and Phanuphak, Nittaya and Fletcher, James L K and Prueksakaew, Peeriya and Jagodzinski, Linda L and Valcour, Victor and Robb, Merlin and Ananworanich, Jintanat and Allen, Isabel E and Krebs, Shelly J and Spudich, Serena}, year={2019}, month=nov, pages={1885–1891}, language={en} }
 @article{Kamarulzaman_Altice_2015, title={Challenges in managing HIV in people who use drugs}, volume={28}, ISSN={0951-7375}, DOI={10.1097/QCO.0000000000000125}, number={1}, journal={Current Opinion in Infectious Diseases}, author={Kamarulzaman, Adeeba and Altice, Frederick L.}, year={2015}, month=feb, pages={10–16}, language={en} }
 @article{Kamitani_Higa_Crepaz_Wichser_Mullins_The U.S. Centers for Disease Control and Prevention’s Prevention Research Synthesis Project_2024, title={Identifying Best Practices for Increasing HIV Pre-exposure Prophylaxis (PrEP) Use and Persistence in the United States: A Systematic Review}, volume={28}, ISSN={1090-7165, 1573-3254}, DOI={10.1007/s10461-024-04332-z}, number={7}, journal={AIDS and Behavior}, author={Kamitani, Emiko and Higa, Darrel H. and Crepaz, Nicole and Wichser, Megan and Mullins, Mary M. and The U.S. Centers for Disease Control and Prevention’s Prevention Research Synthesis Project}, year={2024}, month=july, pages={2340–2349}, language={en} }
 @article{Kamitani_Koenig_Sullivan_2025, title={Transformative potential of artificial intelligence in US CDC HIV interventions: balancing innovation with health privacy}, volume={39}, ISSN={0269-9370, 1473-5571}, DOI={10.1097/QAD.0000000000004220}, abstractNote={Artificial intelligence (AI) holds significant potential to transform HIV prevention and treatment through the application of advanced technologies such as machine learning (ML), deep learning (DL), and generative AI (Gen AI). These technologies can enhance the monitoring, management, and analysis of vast and complex HIV-related datasets, enabling more timely predictions of potential risks and improving HIV care strategies. AI is poised to streamline HIV prevention interventions by increasing workforce efficiency, supporting expanded accessibility and sustainability of preexposure prophylaxis (PrEP) care in nontraditional settings, and supporting clinical decision-making. Additionally, when utilized within HIV care systems, AI can help close gaps in diagnosis, treatment, and continuous care engagement. However, to optimize AI’s potential in HIV prevention, careful implementation is crucial. Challenges such as reducing bias, ensuring ethical standards (including health privacy standards) are maintained, and mitigating risks like AI hallucinations must be addressed. Thoughtful integration, community consultation, and continuous evaluation will be critical to ensuring that AI plays a beneficial role in HIV prevention and drives innovations that lead to more equitable health outcomes. This editorial review explores AI’s transformative potential, focusing on the US CDC’s key public health strategies for HIV prevention. When aligning with public health strategies – particularly in countries supported by initiatives like President’s Emergency Plan for AIDS Relief (PEPFAR) – AI can contribute significantly to global efforts to end the HIV epidemic. It offers a vision for AI’s future application in HIV prevention, emphasizing the need for a holistic and syndemic approach to improving HIV prevention worldwide.}, number={10}, journal={AIDS}, author={Kamitani, Emiko and Koenig, Linda J. and Sullivan, Patrick}, year={2025}, month=aug, pages={1311–1321}, language={en} }
 @article{Kelley_Acevedo-Quiñones_Agwu_Avihingsanon_Benson_Blumenthal_Brinson_Brites_Cahn_Cantos_et al._2025, title={Twice-Yearly Lenacapavir for HIV Prevention in Men and Gender-Diverse Persons}, volume={392}, rights={http://www.nejmgroup.org/legal/terms-of-use.htm}, ISSN={0028-4793, 1533-4406}, DOI={10.1056/NEJMoa2411858}, number={13}, journal={New England Journal of Medicine}, author={Kelley, Colleen F. and Acevedo-Quiñones, Maribel and Agwu, Allison L. and Avihingsanon, Anchalee and Benson, Paul and Blumenthal, Jill and Brinson, Cynthia and Brites, Carlos and Cahn, Pedro and Cantos, Valeria D. and Clark, Jesse and Clement, Meredith and Creticos, Cathy and Crofoot, Gordon and Diaz, Ricardo S. and Doblecki-Lewis, Susanne and Gallardo-Cartagena, Jorge A. and Gaur, Aditya and Grinsztejn, Beatriz and Hassler, Shawn and Hinojosa, Juan Carlos and Hodge, Theo and Kaplan, Richard and Lacerda, Marcus and LaMarca, Anthony and Losso, Marcelo H. and Valdez Madruga, José and Mayer, Kenneth H. and Mills, Anthony and Mounzer, Karam and Ndlovu, Nkosiphile and Novak, Richard M. and Perez Rios, Alma and Phanuphak, Nittaya and Ramgopal, Moti and Ruane, Peter J. and Sánchez, Jorge and Santos, Breno and Schine, Patric and Schreibman, Tanya and Spencer, LaShonda Y. and Van Gerwen, Olivia T. and Vasconcelos, Ricardo and Vasquez, Jose Gabriel and Zwane, Zwelethu and Cox, Stephanie and Deaton, Chris and Ebrahimi, Ramin and Wong, Pamela and Singh, Renu and Brown, Lillian B. and Carter, Christoph C. and Das, Moupali and Baeten, Jared M. and Ogbuagu, Onyema}, year={2025}, month=apr, pages={1261–1276}, language={en} }
 @article{Khan_Dombrowski_Saad_McLean_Friedman_2013, title={Network Firewall Dynamics and the Subsaturation Stabilization of HIV}, volume={2013}, rights={http://creativecommons.org/licenses/by/3.0/}, ISSN={1026-0226, 1607-887X}, DOI={10.1155/2013/720818}, abstractNote={In 2001, Friedman et al. conjectured the existence of a “firewall effect” in which individuals who are infected with HIV, but remain in a state of low infectiousness, serve to prevent the virus from spreading. To evaluate this historical conjecture, we develop a new graph-theoretic measure that quantifies the extent to which Friedman’s firewall hypothesis (FH) holds in a risk network. We compute this new measure across simulated trajectories of a stochastic discrete dynamical system that models a social network of 25,000 individuals engaging in risk acts over a period of 15 years. The model’s parameters are based on analyses of data collected in prior studies of the real-world risk networks of people who inject drugs (PWID) in New York City. Analysis of system trajectories reveals the structural mechanisms by which individuals with mature HIV infections tend to partition the network into homogeneous clusters (with respect to infection status) and how uninfected clusters remain relatively stable (with respect to infection status) over long stretches of time. We confirm the spontaneous emergence of network firewalls in the system and reveal their structural role in the nonspreading of HIV.}, journal={Discrete Dynamics in Nature and Society}, author={Khan, Bilal and Dombrowski, Kirk and Saad, Mohamed and McLean, Katherine and Friedman, Samuel}, year={2013}, pages={1–16}, language={en} }
 @article{Lancaster_Endres-Dighe_Sucaldito_Piscalko_Madhu_Kiriazova_Batchelder_2022, title={Measuring and Addressing Stigma Within HIV Interventions for People Who Use Drugs: a Scoping Review of Recent Research}, volume={19}, ISSN={1548-3576}, DOI={10.1007/s11904-022-00619-9}, abstractNote={PURPOSE OF REVIEW: Persistent stigma remains a crucial barrier to HIV prevention and treatment services among people who use drugs (PWUD), particularly for those living with or at-risk for HIV. This scoping review examines the current state of science with regard to approaches for measuring and addressing stigma within HIV interventions among PWUD.
RECENT FINDINGS: Sixteen studies fit the inclusion criteria for this review. Half the studies originated within the USA, and the remaining represented four different regions. Within these studies, stigma was measured using various quantitative, qualitative, and mixed methods. The studies primarily focused on HIV stigma, including value-based judgments, anticipated stigma, and perceived stigma domains. Information-based and skills building approaches at the individual level were the most common for the stigma reduction interventions. Adoption of systematic evaluations is needed for measuring stigma, including intersectional stigma, within HIV interventions among PWUD. Future studies should focus on developing multilevel intersectional stigma reduction interventions for PWUD with and at-risk for HIV globally.}, number={5}, journal={Current HIV/AIDS reports}, author={Lancaster, Kathryn E. and Endres-Dighe, Stacy and Sucaldito, Ana D. and Piscalko, Hannah and Madhu, Aarti and Kiriazova, Tetiana and Batchelder, Abigail W.}, year={2022}, month=oct, pages={301–311}, language={eng} }
 @article{Landovitz_CAB_2021, title={Cabotegravir for HIV prevention in cisgender men and transgender women}, volume={385}, DOI={10.1056/NEJMoa2101016}, journal={N. Engl. J. Med.}, author={Landovitz, R. J. and Donnell, D. and Clement, M. E. and Hanscom, B. and Cottle, L. and Coelho, L. and Cabello, R. and Chariyalertsak, S. and Dunne, E. F. and Frank, I. and others}, year={2021}, pages={595–608} }
 @article{Liang_Greenhalgh_Mao_2016, title={A Stochastic Differential Equation Model for the Spread of HIV amongst People Who Inject Drugs}, volume={2016}, rights={http://creativecommons.org/licenses/by/4.0/}, ISSN={1748-670X, 1748-6718}, DOI={10.1155/2016/6757928}, abstractNote={We introduce stochasticity into the deterministic differential equation model for the spread of HIV amongst people who inject drugs (PWIDs) studied by Greenhalgh and Hay (1997). This was based on the original model constructed by Kaplan (1989) which analyses the behaviour of HIV/AIDS amongst a population of PWIDs. We derive a stochastic differential equation (SDE) for the fraction of PWIDs who are infected with HIV at time. The stochasticity is introduced using the well-known standard technique of parameter perturbation. We first prove that the resulting SDE for the fraction of infected PWIDs has a unique solution in (0, 1) provided that some infected PWIDs are initially present and next construct the conditions required for extinction and persistence. Furthermore, we show that there exists a stationary distribution for the persistence case. Simulations using realistic parameter values are then constructed to illustrate and support our theoretical results. Our results provide new insight into the spread of HIV amongst PWIDs. The results show that the introduction of stochastic noise into a model for the spread of HIV amongst PWIDs can cause the disease to die out in scenarios where deterministic models predict disease persistence.}, journal={Computational and Mathematical Methods in Medicine}, author={Liang, Yanfeng and Greenhalgh, David and Mao, Xuerong}, year={2016}, pages={1–14}, language={en} }
 @article{Longino_Paul_Wang_Lama_Brandes_Ruiz_Correa_Keating_Spudich_Pilcher_et al._2022, title={HIV Disease Dynamics and Markers of Inflammation and CNS Injury During Primary HIV Infection and Their Relationship to Cognitive Performance}, volume={89}, ISSN={1525-4135}, DOI={10.1097/QAI.0000000000002832}, abstractNote={Introduction:
              Early systemic and central nervous system viral replication and inflammation may affect brain integrity in people with HIV, leading to chronic cognitive symptoms not fully reversed by antiretroviral therapy (ART). This study examined associations between cognitive performance and markers of CNS injury associated with acute HIV infection and ART.
            
            
              Methods:
              HIV-infected MSM and transgender women (average age: 27 years and education: 13 years) enrolled within 100 days from the estimated date of detectable infection (EDDI). A cognitive performance (NP) protocol was administered at enrollment (before ART initiation) and every 24 weeks until week 192. An overall index of cognitive performance (NPZ) was created using local normative data. Blood (n = 87) and cerebrospinal fluid (CSF; n = 29) biomarkers of inflammation and neuronal injury were examined before ART initiation. Regression analyses assessed relationships between time since EDDI, pre-ART biomarkers, and NPZ.
            
            
              Results:
              Adjusting for multiple comparisons, shorter time since EDDI was associated with higher pre-ART VL and multiple biomarkers in plasma and CSF. NPZ scores were within the normative range at baseline (NPZ = 0.52) and at each follow-up visit, with a modest increase through week 192. Plasma or CSF biomarkers were not correlated with NP scores at baseline or after ART.
            
            
              Conclusions:
              Biomarkers of CNS inflammation, immune activation, and neuronal injury peak early and then decline during acute HIV infection, confirming and extending results of other studies. Neither plasma nor CSF biomarkers during acute infection corresponded to NP scores before or after sustained ART in this cohort with few psychosocial risk factors for cognitive impairment.}, number={2}, journal={JAIDS Journal of Acquired Immune Deficiency Syndromes}, author={Longino, August A. and Paul, Robert and Wang, Yixin and Lama, Javier R. and Brandes, Peter and Ruiz, Eduardo and Correa, Cecilia and Keating, Sheila and Spudich, Serena S. and Pilcher, Christopher and Vecchio, Alyssa and Pasalar, Siavash and Bender Ignacio, Rachel A. and Valdez, Rogelio and Dasgupta, Sayan and Robertson, Kevin and Duerr, Ann}, year={2022}, month=feb, pages={183–190}, language={en} }
 @article{Marzinke_Fogel_Wang_Piwowar-Manning_Kofron_Moser_Bhandari_Gollings_Bushman_Weng_et al._2023, title={Extended Analysis of HIV Infection in Cisgender Men and Transgender Women Who Have Sex with Men Receiving Injectable Cabotegravir for HIV Prevention: HPTN 083}, volume={67}, ISSN={0066-4804, 1098-6596}, DOI={10.1128/aac.00053-23}, abstractNote={HPTN 083 demonstrated that injectable cabotegravir (CAB) was superior to oral tenofovir disoproxil fumarate-emtricitabine (TDF-FTC) for HIV prevention in cisgender men and transgender women who have sex with men. We previously analyzed 58 infections in the blinded phase of HPTN 083 (16 in the CAB arm and 42 in the TDF-FTC arm).
          , 
            ABSTRACT
            HPTN 083 demonstrated that injectable cabotegravir (CAB) was superior to oral tenofovir disoproxil fumarate-emtricitabine (TDF-FTC) for HIV prevention in cisgender men and transgender women who have sex with men. We previously analyzed 58 infections in the blinded phase of HPTN 083 (16 in the CAB arm and 42 in the TDF-FTC arm). This report describes 52 additional infections that occurred up to 1 year after study unblinding (18 in the CAB arm and 34 in the TDF-FTC arm). Retrospective testing included HIV testing, viral load testing, quantification of study drug concentrations, and drug resistance testing. The new CAB arm infections included 7 with CAB administration within 6 months of the first HIV-positive visit (2 with on-time injections, 3 with ≥1 delayed injection, and 2 who restarted CAB) and 11 with no recent CAB administration. Three cases had integrase strand transfer inhibitor (INSTI) resistance (2 with on-time injections and 1 who restarted CAB). Among 34 CAB infections analyzed to date, diagnosis delays and INSTI resistance were significantly more common in infections with CAB administration within 6 months of the first HIV-positive visit. This report further characterizes HIV infections in persons receiving CAB preexposure prophylaxis and helps define the impact of CAB on the detection of infection and the emergence of INSTI resistance.}, number={4}, journal={Antimicrobial Agents and Chemotherapy}, author={Marzinke, Mark A. and Fogel, Jessica M. and Wang, Zhe and Piwowar-Manning, Estelle and Kofron, Ryan and Moser, Amber and Bhandari, Pradip and Gollings, Ryann and Bushman, Lane R. and Weng, Lei and Halvas, Elias K. and Mellors, John and Anderson, Peter L. and Persaud, Deborah and Hendrix, Craig W. and McCauley, Marybeth and Rinehart, Alex R. and St Clair, Marty and Ford, Susan L. and Rooney, James F. and Adeyeye, Adeola and Chariyalertsak, Suwat and Mayer, Kenneth and Arduino, Roberto C. and Cohen, Myron S. and Grinsztejn, Beatriz and Hanscom, Brett and Landovitz, Raphael J. and Eshleman, Susan H.}, year={2023}, month=apr, pages={e00053-23}, language={en} }
 @article{Mistler_Copenhaver_Shrestha_2021, title={The Pre-exposure Prophylaxis (PrEP) Care Cascade in People Who Inject Drugs: A Systematic Review}, volume={25}, ISSN={1090-7165, 1573-3254}, DOI={10.1007/s10461-020-02988-x}, number={5}, journal={AIDS and Behavior}, author={Mistler, Colleen B. and Copenhaver, Michael M. and Shrestha, Roman}, year={2021}, month=may, pages={1490–1506}, language={en} }
 @article{Muncan_Walters_Ezell_Ompad_2020, title={“They look at us like junkies”: influences of drug use stigma on the healthcare engagement of people who inject drugs in New York City}, volume={17}, ISSN={1477-7517}, DOI={10.1186/s12954-020-00399-8}, abstractNote={Abstract
            
              Background
              People who inject drugs (PWID) are a medically and socially vulnerable population with a high incidence of overdose, mental illness, and infections like HIV and hepatitis C. Existing literature describes social and economic correlations to increased health risk, including stigma. Injection drug use stigma has been identified as a major contributor to healthcare disparities for PWID. However, data on this topic, particularly in terms of the interface between enacted, anticipated, and internalized stigma, is still limited. To fill this gap, we examined perspectives from PWID whose stigmatizing experiences impacted their views of the healthcare system and syringe service programs (SSPs) and influenced their decisions regarding future medical care.
            
            
              Methods
              Semi-structured interviews conducted with 32 self-identified PWID in New York City. Interviews were audio recorded and transcribed. Interview transcripts were coded using a grounded theory approach by three trained coders and key themes were identified as they emerged.
            
            
              Results
              A total of 25 participants (78.1%) reported at least one instance of stigma related to healthcare system engagement. Twenty-three participants (71.9%) reported some form of enacted stigma with healthcare, 19 participants (59.4%) described anticipated stigma with healthcare, and 20 participants (62.5%) reported positive experiences at SSPs. Participants attributed healthcare stigma to their drug injection use status and overwhelmingly felt distrustful of, and frustrated with, medical providers and other healthcare staff at hospitals and local clinics. PWID did not report internalized stigma, in part due to the availability of non-stigmatizing medical care at SSPs.
            
            
              Conclusions
              Stigmatizing experiences of PWID in formal healthcare settings contributed to negative attitudes toward seeking healthcare in the future. Many participants describe SSPs as accessible sites to receive high-quality medical care, which may curb the manifestation of internalized stigma derived from negative experiences in the broader healthcare system. Our findings align with those reported in the literature and reveal the potentially important role of SSPs. With the goal of limiting stigmatizing interactions and their consequences on PWID health, we recommend that future research include explorations of mechanisms by which PWID make decisions in stigmatizing healthcare settings, as well as improving medical care availability at SSPs.}, number={1}, journal={Harm Reduction Journal}, author={Muncan, Brandon and Walters, Suzan M. and Ezell, Jerel and Ompad, Danielle C.}, year={2020}, month=dec, pages={53}, language={en} }
 @article{Patel_Hoover_Lale_Cabrales_Byrd_Kourtis_2025, title={Clinical Recommendation for the Use of Injectable Lenacapavir as HIV Preexposure Prophylaxis — United States, 2025}, volume={74}, ISSN={0149-2195, 1545-861X}, DOI={10.15585/mmwr.mm7435a1}, number={35}, journal={MMWR. Morbidity and Mortality Weekly Report}, author={Patel, Rupa R. and Hoover, Karen W. and Lale, Allison and Cabrales, Janet and Byrd, Katrina M. and Kourtis, Athena P.}, year={2025}, month=sept, pages={541–549} }
 @article{Pridgen_Bontemps_Lloyd_Wagner_Kay_Eaton_Cropsey_2025, title={U.S. substance use harm reduction efforts: a review of the current state of policy, policy barriers, and recommendations}, volume={22}, ISSN={1477-7517}, DOI={10.1186/s12954-025-01238-4}, number={1}, journal={Harm Reduction Journal}, author={Pridgen, Bailey E. and Bontemps, Andrew P. and Lloyd, Audrey R. and Wagner, William P. and Kay, Emma S. and Eaton, Ellen F. and Cropsey, Karen L.}, year={2025}, month=june, pages={101}, language={en} }
 @article{Randall_Dasgupta_Day_DeMaria_Musolino_John_Cranston_Buchacz_2022, title={An outbreak of HIV infection among people who inject drugs in northeastern Massachusetts: findings and lessons learned from a medical record review}, volume={22}, ISSN={1471-2458}, DOI={10.1186/s12889-022-12604-3}, abstractNote={Abstract
            
              Background
              We conducted a medical record review for healthcare utilization, risk factors, and clinical data among people who inject drugs (PWID) in Massachusetts to aid HIV outbreak response decision-making and strengthen public health practice.
            
            
              Setting
              Two large community health centers (CHCs) that provide HIV and related services in northeastern Massachusetts.
            
            
              Methods
              Between May and July 2018, we reviewed medical records for 88 people with HIV (PWH) connected to the outbreak. The review period included care received from May 1, 2016, through the date of review. Surveillance data were used to establish date of HIV diagnosis and assess viral suppression.
            
            
              Results
              Sixty-nine (78%) people had HIV infection diagnosed during the review period, including 10 acute infections. Persons had a median of 3 primary care visits after HIV diagnosis and zero before diagnosis. During the review period, 72% reported active drug or alcohol use, 62% were prescribed medication assisted treatment, and 41% were prescribed antidepressants. The majority (68, 77%) had a documented ART prescription. HIV viral suppression at < 200 copies/mL was more frequent (73%) than the overall across the State (65%); it did not correlate with any of the sociodemographic characteristics studied in our population. Over half (57%) had been hospitalized at least once during the review period, and 36% had a bacterial infection at hospitalization.
            
            
              Conclusions
              Medical record review with a field investigation of an outbreak provided data about patterns of health care utilization and comorbidities not available from routine HIV surveillance or case interviews. Integration of HIV screening with treatment for HIV and SUD can strengthen prevention and care services for PWID in northeastern Massachusetts.}, number={1}, journal={BMC Public Health}, author={Randall, Liisa M. and Dasgupta, Sharoda and Day, Jeanne and DeMaria, Alfred and Musolino, Joseph and John, Betsey and Cranston, Kevin and Buchacz, Kate}, year={2022}, month=dec, pages={257}, language={en} }
 @article{Reid_Guthrie_Hajat_Glick_2025, title={National trends in co-use of opioids and methamphetamine among people who inject drugs, 2012–2018}, volume={271}, ISSN={03768716}, DOI={10.1016/j.drugalcdep.2025.112630}, journal={Drug and Alcohol Dependence}, author={Reid, Molly C. and Guthrie, Brandon L. and Hajat, Anjum and Glick, Sara N.}, year={2025}, month=june, pages={112630}, language={en} }
 @article{Rodger_Cambiano_Bruun_Vernazza_Collins_Degen_Corbelli_Estrada_Geretti_Beloukas_et al._2019, title={Risk of HIV transmission through condomless sex in serodifferent gay couples with the HIV-positive partner taking suppressive antiretroviral therapy (PARTNER): final results of a multicentre, prospective, observational study}, volume={393}, ISSN={01406736}, DOI={10.1016/S0140-6736(19)30418-0}, number={10189}, journal={The Lancet}, author={Rodger, Alison J and Cambiano, Valentina and Bruun, Tina and Vernazza, Pietro and Collins, Simon and Degen, Olaf and Corbelli, Giulio Maria and Estrada, Vicente and Geretti, Anna Maria and Beloukas, Apostolos and Raben, Dorthe and Coll, Pep and Antinori, Andrea and Nwokolo, Nneka and Rieger, Armin and Prins, Jan M and Blaxhult, Anders and Weber, Rainer and Van Eeden, Arne and Brockmeyer, Norbert H and Clarke, Amanda and Del Romero Guerrero, Jorge and Raffi, Francois and Bogner, Johannes R and Wandeler, Gilles and Gerstoft, Jan and Gutiérrez, Felix and Brinkman, Kees and Kitchen, Maria and Ostergaard, Lars and Leon, Agathe and Ristola, Matti and Jessen, Heiko and Stellbrink, Hans-Jürgen and Phillips, Andrew N and Lundgren, Jens and Coll, Pep and Cobarsi, Patricia and Nieto, Aroa and Meulbroek, Michael and Carrillo, Antonia and Saz, Jorge and Guerrero, Jorge D.R. and García, Mar Vera and Gutiérrez, Felix and Masiá, Mar and Robledano, Catalina and Leon, Agathe and Leal, Lorna and Redondo, Eva G. and Estrada, Vicente P. and Marquez, Rocio and Sandoval, Raquel and Viciana, Pompeyo and Espinosa, Nuria and Lopez-Cortes, Luis and Podzamczer, Daniel and Tiraboschi, Juan and Morenilla, Sandra and Antela, Antonio and Losada, Elena and Nwokolo, Nneka and Sewell, Janey and Clarke, Amanda and Kirk, Sarah and Knott, Alyson and Rodger, Alison J and Fernandez, Thomas and Gompels, Mark and Jennings, Louise and Ward, Lana and Fox, Julie and Lwanga, Julianne and Lee, Ming and Gilson, Richard and Leen, Clifford and Morris, Sheila and Clutterbuck, Dan and Brady, Michael and Asboe, David and Fedele, Serge and Fidler, Sarah and Brockmeyer, Norbert and Potthoff, Anja and Skaletz-Rorowski, Adriane and Bogner, Johannes and Seybold, Ulrich and Roider, Julia and Jessen, Heiko and Jessen, Arne and Ruzicic, Slobodan and Stellbrink, Hans-Jürgen and Kümmerle, Tim and Lehmann, Clara and Degen, Olaf and Bartel, Sindy and Hüfner, Anja and Rockstroh, Jürgen and Mohrmann, Karina and Boesecke, Christoph and Krznaric, Ivanka and Ingiliz, Patrick and Weber, Rainer and Grube, Christina and Braun, Dominique and Günthard, Huldrych and Wandeler, Gilles and Furrer, Hansjakob and Rauch, Andri and Vernazza, Pietro and Schmid, Patrick and Rasi, Manuela and Borso, Denise and Stratmann, Markus and Caviezel, Oliver and Stoeckle, Marcel and Battegay, Manuel and Tarr, Philip and Christinet, Vanessa and Jouinot, Florent and Isambert, Camille and Bernasconi, Enos and Bernasconi, Beatrice and Gerstoft, Jan and Jensen, Lene P. and Bayer, Anne A. and Ostergaard, Lars and Yehdego, Yordanos and Bach, Ann and Handberg, Pia and Kronborg, Gitte and Pedersen, Svend S. and Bülow, Nete and Ramskover, Bente and Ristola, Matti and Debnam, Outi and Sutinen, Jussi and Blaxhult, Anders and Ask, Ronnie and Hildingsson-Lundh, Bernt and Westling, Katarina and Frisen, Eeva-Maija and Cortney, Gráinne and O’Dea, Siobhan and De Wit, Stephane and Necsoi, Coca and Vandekerckhove, Linos and Goffard, Jean-Christophe and Henrard, Sophie and Prins, Jan and Nobel, Hans-Henrik and Weijsenfeld, Annouschka and Van Eeden, Arne and Elsenburg, Loek and Brinkman, Kees and Vos, Danielle and Hoijenga, Imke and Gisolf, Elisabeth and Van Bentum, Petra and Verhagen, Dominique and Raffi, Francois and Billaud, Eric and Ohayon, Michel and Gosset, Daniel and Fior, Alexandre and Pialoux, Gilles and Thibaut, Pelagie and Chas, Julie and Leclercq, Vincent and Pechenot, Vincent and Coquelin, Vincent and Pradier, Christian and Breaud, Sophie and Touzeau-Romer, Veronique and Rieger, Armin and Kitchen - Maria Geit, Maria and Sarcletti, Mario and Gisinger, Martin and Oellinger, Angela and Antinori, Andrea and Menichetti, Samanta and Bini, Teresa and Mussini, Cristina and Meschiari, Marianna and Di Biagio, Antonio and Taramasso, Lucia and Celesia, Benedetto M. and Gussio, Maria and Janeiro, Nuno}, year={2019}, month=june, pages={2428–2438}, language={en} }
 @article{Rozansky_Christine_Younkin_Fox_Weinstein_Suarez_Stewart_Farrell_Taylor_2024, title={Addiction consult service involvement in PrEP and PEP delivery for patients who inject drugs admitted to an urban essential hospital}, volume={19}, ISSN={1940-0640}, DOI={10.1186/s13722-024-00502-5}, abstractNote={BACKGROUND: Addiction medicine providers have a key role in HIV prevention amidst rising HIV incidence in persons who inject drugs (PWID). Pre-exposure prophylaxis (PrEP) and post-exposure prophylaxis (PEP) are vastly underutilized in this population. Inpatient hospitalization represents a potential touchpoint for initiation of HIV prophylaxis, though little research explores the role of addiction providers. Here we describe rates of PrEP/PEP delivery to hospitalized PWID seen by an Addiction Consult Service (ACS) at an urban, essential hospital.
METHODS: We performed a cross-sectional study of hospitalized patients who were seen by the ACS from January 1, 2020 to December 31, 2022 and had plausible injection drug use. We calculated the proportion of patients who received a new prescription for PrEP/PEP at discharge. We used descriptive statistics to characterize demographics, substance use, reason for admission, and indications for PrEP/PEP. Secondarily, we calculated the monthly proportion of all patients discharged from the hospital with PrEP/PEP who were seen by the ACS compared to those not seen by the ACS.
RESULTS: The average monthly proportion of ACS consults with plausible injection drug use who received PrEP/PEP was 6.4%. This increased from 4.2% in 2020 to 7.5% in 2022. Those seen by the ACS who received PrEP/PEP had high rates of opioid use disorder (97.5%), stimulant use disorder (77.8%), and homelessness (58.1%); over half were admitted for an injection-related infection. The indications for PrEP/PEP were injection drug use only (70.6%), followed by combined injection and sexual risk (20.2%); 71.9% of prescriptions were for PrEP and 28.1% for PEP. Overall, the ACS was involved in 83.9% of hospital-wide discharges with PrEP/PEP prescriptions (n = 242).
CONCLUSIONS: PWID who were seen by the ACS received PrEP/PEP prescriptions at rates exceeding national averages. The ACS was also involved with the care of the majority of admitted patients who received PrEP/PEP at discharge. While PrEP/PEP use for PWID remains low, the inpatient ACS represents a key resource to improve uptake by leveraging the reachable moment of an inpatient hospitalization.}, number={1}, journal={Addiction Science & Clinical Practice}, author={Rozansky, Hallie and Christine, Paul J. and Younkin, Morgan and Fox, Jason M. and Weinstein, Zoe M. and Suarez, Sebastian and Stewart, Jessica and Farrell, Natalija and Taylor, Jessica L.}, year={2024}, month=nov, pages={77}, language={eng} }
 @article{Rutstein_Sellers_Ananworanich_Cohen_2015b, title={The HIV treatment cascade in acutely infected people: informing global guidelines}, volume={10}, ISSN={1746-630X}, DOI={10.1097/COH.0000000000000193}, number={6}, journal={Current Opinion in HIV and AIDS}, author={Rutstein, Sarah E. and Sellers, Christopher J. and Ananworanich, Jintanat and Cohen, Myron S.}, year={2015}, month=nov, pages={395–402}, language={en} }
 @article{Simmons_McClean_2014, title={Brain cell injury in HIV infection: When does it start and can it be stopped?}, volume={83}, ISSN={0028-3878, 1526-632X}, url={https://www.neurology.org/doi/10.1212/WNL.0000000000001016}, DOI={10.1212/WNL.0000000000001016}, number={18}, journal={Neurology}, author={Simmons, Daniel B. and McClean, Jeffrey C.}, year={2014}, month=oct, language={en} }
 @article{Stone_Fraser_Lim_Walker_Ward_MacGregor_Trickey_Abbott_Strathdee_Abramovitz_et al._2018, title={Incarceration history and risk of HIV and hepatitis C virus acquisition among people who inject drugs: a systematic review and meta-analysis}, volume={18}, ISSN={14733099}, DOI={10.1016/S1473-3099(18)30469-9}, number={12}, journal={The Lancet Infectious Diseases}, author={Stone, Jack and Fraser, Hannah and Lim, Aaron G and Walker, Josephine G and Ward, Zoe and MacGregor, Louis and Trickey, Adam and Abbott, Sam and Strathdee, Steffanie A and Abramovitz, Daniela and Maher, Lisa and Iversen, Jenny and Bruneau, Julie and Zang, Geng and Garfein, Richard S and Yen, Yung-Fen and Azim, Tasnim and Mehta, Shruti H and Milloy, Michael-John and Hellard, Margaret E and Sacks-Davis, Rachel and Dietze, Paul M and Aitken, Campbell and Aladashvili, Malvina and Tsertsvadze, Tengiz and Mravčík, Viktor and Alary, Michel and Roy, Elise and Smyrnov, Pavlo and Sazonova, Yana and Young, April M and Havens, Jennifer R and Hope, Vivian D and Desai, Monica and Heinsbroek, Ellen and Hutchinson, Sharon J and Palmateer, Norah E and McAuley, Andrew and Platt, Lucy and Martin, Natasha K and Altice, Frederick L and Hickman, Matthew and Vickerman, Peter}, year={2018}, month=dec, pages={1397–1409}, language={en} }
 @article{Strathdee_Kuo_El-Bassel_Hodder_Smith_Springer_2020, title={Preventing HIV outbreaks among people who inject drugs in the United States: plus ça change, plus ça même chose}, volume={34}, ISSN={0269-9370, 1473-5571}, DOI={10.1097/QAD.0000000000002673}, abstractNote={This editorial review covers current trends in the epidemiology of HIV among people who inject drugs (PWID) in the United States, including four recent HIV outbreaks. We discuss gaps in the prevention and treatment cascades for HIV and medications for opioid disorder and propose lessons learned to prevent future HIV outbreaks. Over the last decade, North America has been in the throes of a major opioid epidemic, due in part to over-prescribing of prescription opiates, followed by increasing availability of cheap heroin, synthetic opioids (e.g. fentanyl), and stimulants (e.g. methamphetamine). Historically, HIV infection among PWID in the US had predominantly affected communities who were older, urban and Black. More recently, the majority of these infections are among younger, rural or suburban and Caucasian PWID. All four HIV outbreaks were characterized by a high proportion of women who inject drugs and underlying socioeconomic drivers such as homelessness and poverty. We contend that the US response to the HIV epidemic among PWID has been fractured. A crucial lesson is that when evidence-based responses to HIV prevention are undermined or abandoned because of moral objections, untold humanitarian and financial costs on public health will ensue. Restructuring a path forward requires that evidence-based interventions be integrated and brought to scale while simultaneously addressing underlying structural drivers of HIV and related syndemics. Failing to do so will mean that HIV outbreaks among PWID and the communities they live in will continue to occur in a tragic and relentless cycle.}, number={14}, journal={AIDS}, author={Strathdee, Steffanie A. and Kuo, Irene and El-Bassel, Nabila and Hodder, Sally and Smith, Laramie R. and Springer, Sandra A.}, year={2020}, month=nov, pages={1997–2005}, language={en} }
 @article{Sued_Nardi_Spadaccini_2022, title={Key population perceptions and opinions about long-acting antiretrovirals for prevention and treatment: a scoping review}, volume={17}, ISSN={1746-6318}, DOI={10.1097/COH.0000000000000734}, abstractNote={PURPOSE OF REVIEW: Key populations are disproportionately affected by human immunodeficiency virus (HIV). Access, retention, and adherence are important barriers for the efficacy of preexposure prophylaxis (PrEP) and HIV treatment among these populations. Long-acting (LA) antiretrovirals hold the promise to solve some of these backdrops. The objective of the current review is to update the perceptions of key populations and PLWH about LA, based on their opinion, acceptability, and willingness to use it.
RECENT FINDINGS: According to the review preferences for LA vary with the population studied. Regarding people living with HIV (PLWH), male having sex with men are interested in having different options, adolescents are interested in LA (strong preference for implants), yet also perceive substantial obstacles to using biomedical prevention; transgender women aimed to nonvisible small implants, with long-lasting effects or LA injections that can be applied in other areas than buttocks, and women who experienced history of medical injections might increase preference for LA (except for history of people who inject drugs [IDU]). Female sex workers and IDU both showed interest in LA-PrEP. Regarding antiretroviral therapy, LA increased treatment satisfaction and acceptance, mainly among those receiving injections every 2 months. LA helped overcome pill fatigue, stigma, and adherence issues.
SUMMARY: Knowing preferences for biomedical interventions will contribute to better understanding and developing effective strategies for these populations.}, number={3}, journal={Current opinion in HIV and AIDS}, author={Sued, Omar and Nardi, Norma and Spadaccini, Luciana}, year={2022}, month=may, pages={145–161}, language={eng} }
 @article{Surratt_Otachi_McLouth_Vundi_2021, title={Healthcare stigma and HIV risk among rural people who inject drugs}, volume={226}, ISSN={03768716}, DOI={10.1016/j.drugalcdep.2021.108878}, journal={Drug and Alcohol Dependence}, author={Surratt, Hilary L. and Otachi, Janet K. and McLouth, Christopher J. and Vundi, Nikita}, year={2021}, month=sept, pages={108878}, language={en} }
 @article{Tanner_O’Shea_Byrd_Johnston_Dumitru_Le_Lale_Byrd_Cholli_Kamitani_et al._2025, title={Antiretroviral Postexposure Prophylaxis After Sexual, Injection Drug Use, or Other Nonoccupational Exposure to HIV — CDC Recommendations, United States, 2025}, volume={74}, ISSN={1057-5987, 1545-8601}, DOI={10.15585/mmwr.rr7401a1}, number={1}, journal={MMWR. Recommendations and Reports}, author={Tanner, Mary R. and O’Shea, Jesse G. and Byrd, Katrina M. and Johnston, Marie and Dumitru, Gema G. and Le, John N. and Lale, Allison and Byrd, Kathy K. and Cholli, Preetam and Kamitani, Emiko and Zhu, Weiming and Hoover, Karen W. and Kourtis, Athena P.}, year={2025}, month=may, pages={1–56} }
 @article{Taylor_Walley_Bazzi_2019, title={Stuck in the window with you: HIV exposure prophylaxis in the highest risk people who inject drugs}, volume={40}, ISSN={1547-0164}, DOI={10.1080/08897077.2019.1675118}, abstractNote={The opioid and polysubstance epidemics could drive a surge in new HIV infections among people who inject drugs (PWID). Longstanding strategies to reduce HIV incidence, including syringe service programs, condom distribution, medications for opioid use disorder, and low-barrier HIV testing and treatment have not been adequate to eliminate transmission in this population. Although HIV pre-exposure prophylaxis (PrEP) is an evidence-based intervention that reduces HIV incidence among PWID, uptake in PWID has lagged due to limited PrEP knowledge, discrepancies between perceived and actual HIV risk, stigma, and structural barriers to adherence including homelessness and incarceration. In our efforts to deploy PrEP to PWID in our low-barrier substance use disorder bridge clinic, we have encountered another barrier: the HIV testing window period. We discuss challenges in delivering HIV exposure prophylaxis to the highest risk PWID, our current approach, and the need for more data to guide best practices.}, number={4}, journal={Substance Abuse}, author={Taylor, Jessica L. and Walley, Alexander Y. and Bazzi, Angela R.}, year={2019}, pages={441–443}, language={eng} }
 @article{Tookes_Bartholomew_Geary_Matthias_Poschman_Blackmore_Philip_Suarez_Forrest_Rodriguez_et al._2020, title={Rapid Identification and Investigation of an HIV Risk Network Among People Who Inject Drugs –Miami, FL, 2018}, volume={24}, ISSN={1090-7165, 1573-3254}, DOI={10.1007/s10461-019-02680-9}, abstractNote={Abstract
            Prevention of HIV outbreaks among people who inject drugs remains a challenge to ending the HIV epidemic in the United States. The first legal syringe services program (SSP) in Florida implemented routine screening in 2018 leading to the identification of ten anonymous HIV seroconversions. The SSP collaborated with the Department of Health to conduct an epidemiologic investigation. All seven acute HIV seroconversions were linked to care (86% within 30 days) and achieved viral suppression (mean 70 days). Six of the seven individuals are epidemiologically and/or socially linked to at least two other seroconversions. Analysis of the HIV genotypes revealed that two individuals are connected molecularly at 0.5% genetic distance. We identified a risk network with complex transmission dynamics that could not be explained by epidemiological methods or molecular analyses alone. Providing wrap-around services through the SSP, including routine screening, intensive linkage and patient navigation, could be an effective model for achieving viral suppression for people who inject drugs.
          , 
            Resumen
            La prevención de brotes de VIH entre las personas que se inyectan drogas sigue siendo un desafío para poner fin a la epidemia de VIH en los Estados Unidos. El primer programa legal de servicios de intercambio de jeringas (SSP como se conoce con sus siglas en inglés) en Florida, implementó pruebas rutinarias en 2018 que condujo a la identificación de diez seroconversiones anónimas de VIH. El SSP colaboró ​​con el Departamento de Salud para realizar una investigación epidemiológica. Siete seroconversiones agudas de VIH fueron identificadas y se vincularon a la atención médica (86% en 30 días) y lograron la supresión viral (media 70 días). Seis de los siete individuos están vinculados epidemiológicamente y/o socialmente con al menos otras dos seroconversiones. El análisis de los genotipos del VIH reveló que dos individuos están conectados molecularmente a una distancia genética de 0.5%. Identificamos una red de riesgo con una dinámica de transmisión compleja que no puede explicarse solamente con métodos epidemiológicos o análisis moleculares. Brindando servicios integrales a través del SSP, que incluyan pruebas de rutina, el enlace intensivo y la navegación del paciente, podría ser un modelo eficaz para lograr la supresión viral de las personas que se inyectan drogas.}, number={1}, journal={AIDS and Behavior}, author={Tookes, Hansel and Bartholomew, Tyler S. and Geary, Shana and Matthias, James and Poschman, Karalee and Blackmore, Carina and Philip, Celeste and Suarez, Edward and Forrest, David W. and Rodriguez, Allan E. and Kolber, Michael A. and Knaul, Felicia and Colucci, Leah and Spencer, Emma}, year={2020}, month=jan, pages={246–256}, language={en} }
 @article{Van Handel_Rose_Hallisey_Kolling_Zibbell_Lewis_Bohm_Jones_Flanagan_Siddiqi_et al._2016, title={County-Level Vulnerability Assessment for Rapid Dissemination of HIV or HCV Infections Among Persons Who Inject Drugs, United States}, volume={73}, ISSN={1525-4135}, DOI={10.1097/QAI.0000000000001098}, abstractNote={Objective:
              A recent HIV outbreak in a rural network of persons who inject drugs (PWID) underscored the intersection of the expanding epidemics of opioid abuse, unsterile injection drug use (IDU), and associated increases in hepatitis C virus (HCV) infections. We sought to identify US communities potentially vulnerable to rapid spread of HIV, if introduced, and new or continuing high rates of HCV infections among PWID.
            
            
              Design:
              We conducted a multistep analysis to identify indicator variables highly associated with IDU. We then used these indicator values to calculate vulnerability scores for each county to identify which were most vulnerable.
            
            
              Methods:
              We used confirmed cases of acute HCV infection reported to the National Notifiable Disease Surveillance System, 2012–2013, as a proxy outcome for IDU, and 15 county-level indicators available nationally in Poisson regression models to identify indicators associated with higher county acute HCV infection rates. Using these indicators, we calculated composite index scores to rank each county’s vulnerability.
            
            
              Results:
              A parsimonious set of 6 indicators were associated with acute HCV infection rates (proxy for IDU): drug-overdose deaths, prescription opioid sales, per capita income, white, non-Hispanic race/ethnicity, unemployment, and buprenorphine prescribing potential by waiver. Based on these indicators, we identified 220 counties in 26 states within the 95th percentile of most vulnerable.
            
            
              Conclusions:
              Our analysis highlights US counties potentially vulnerable to HIV and HCV infections among PWID in the context of the national opioid epidemic. State and local health departments will need to further explore vulnerability and target interventions to prevent transmission.}, number={3}, journal={JAIDS Journal of Acquired Immune Deficiency Syndromes}, author={Van Handel, Michelle M. and Rose, Charles E. and Hallisey, Elaine J. and Kolling, Jessica L. and Zibbell, Jon E. and Lewis, Brian and Bohm, Michele K. and Jones, Christopher M. and Flanagan, Barry E. and Siddiqi, Azfar-E-Alam and Iqbal, Kashif and Dent, Andrew L. and Mermin, Jonathan H. and McCray, Eugene and Ward, John W. and Brooks, John T.}, year={2016}, month=nov, pages={323–331}, language={en} }
 @article{Wahl_Al-Harthi_2023, title={HIV infection of non-classical cells in the brain}, volume={20}, ISSN={1742-4690}, DOI={10.1186/s12977-023-00616-9}, abstractNote={Abstract HIV-associated neurological disorders (HAND) affect up to 50% of people living with HIV (PLWH), even in the era of combination antiretroviral therapy (cART). HIV-DNA can be detected in the cerebral spinal fluid (CSF) of approximately half of aviremic ART-suppressed PLWH and its presence is associated with poorer neurocognitive performance. HIV DNA + and HIV RNA + cells have also been observed in postmortem brain tissue of individuals with sustained cART suppression. In this review, we provide an overview of how HIV invades the brain and HIV infection of resident brain glial cells (astrocytes and microglia). We also discuss the role of resident glial cells in persistent neuroinflammation and HAND in PLWH and their potential contribution to the HIV reservoir. HIV eradication strategies that target persistently infected glia cells will likely be needed to achieve HIV cure.}, number={1}, journal={Retrovirology}, author={Wahl, Angela and Al-Harthi, Lena}, year={2023}, month=jan, pages={1}, language={en} }
 @article{Walters_Busy_Hamel_Junge_Menza_Mitchell_Pinsent_Toevs_Vines_2022, title={Use of Injection Drugs and Any Form of Methamphetamine in the Portland, OR Metro Area as a Driver of an HIV Time-Space Cluster: Clackamas, Multnomah, and Washington Counties, 2018-2020}, volume={26}, ISSN={1573-3254}, DOI={10.1007/s10461-021-03522-3}, abstractNote={We describe the response to detection of a time-space cluster of new HIV infection in the Portland, OR metro area among people who inject drugs (PWID) and/or people who use any form of methamphetamine. This time-space cluster took place in a region with a syndemic of homelessness and drug use. The investigation included new HIV diagnoses in 2018, 2019, and 2020 in Clackamas, Multnomah, and Washington Counties. Public health response included activating incident command, development and implementation of an enhanced interview tool, outreach testing, and stakeholder engagement. We identified 396 new cases of HIV infection, 116 (29%) of which met the cluster definition. Most cluster cases had no molecular relationships to previous cases. Persons responding to the enhanced interview tool reported behaviors associated with HIV acquisition. Field outreach testing did not identify any new HIV cases but did identify hepatitis C and syphilis infections. We show the importance of a robust public health response to a time-space cluster of new HIV infections in an urban area.}, number={6}, journal={AIDS and behavior}, author={Walters, Jaime and Busy, Lea and Hamel, Christopher and Junge, Kelsi and Menza, Timothy and Mitchell, Jaxon and Pinsent, Taylor and Toevs, Kim and Vines, Jennifer}, year={2022}, month=june, pages={1717–1726}, language={eng} }
 @misc{World Health Organization_2025, title={WHO Recommends Injectable Lenacapavir for HIV Prevention}, url={https://www.who.int/news/item/14-07-2025-who-recommends-injectable-lenacapavir-for-hiv-prevention}, author={World Health Organization}, year={2025}, month=july }
