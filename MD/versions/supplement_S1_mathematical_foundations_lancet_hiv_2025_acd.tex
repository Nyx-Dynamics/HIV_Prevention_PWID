\documentclass[11pt]{article}

\usepackage[margin=1in]{geometry}
\usepackage{times}
\usepackage{amsmath}
\usepackage{amssymb}
\usepackage{booktabs}
\usepackage{hyperref}
\usepackage{setspace}

\onehalfspacing

\title{\textbf{Supplement S1: Mathematical Foundations}\\[0.5cm]
\large The Prevention Theorem, Nested Barrier Framework,\\and Signal-to-Noise Ratio Analysis\\[0.5cm]
\normalsize Supporting the manuscript:\\
\textbf{Structural barriers drive near-zero population-level effectiveness of HIV prevention among people who inject drugs:}\\
\textbf{A Computational Modeling Study}}

\author{AC Demidont, DO\\Independent Researcher, Nyx Dynamics LLC}
\date{December 27, 2025}

\begin{document}

\maketitle

\tableofcontents
\newpage

\noindent\textbf{This supplement provides formal mathematical derivations and methodological foundations underlying the main analysis. All results here are intended to support, not extend, the conclusions of the primary manuscript.}

\section{Epidemiological Foundations}

\subsection{The Basic Reproductive Number}

\textbf{Definition 1.1} (Basic Reproductive Number). The basic reproductive number $R_0$ is defined as the expected number of secondary infections produced by a single infected individual in a completely susceptible population:

\begin{equation}
R_0 = \beta \cdot c \cdot D
\end{equation}

where $\beta$ = transmission probability per contact, $c$ = contact rate, and $D$ = duration of infectiousness.

For injection-related HIV transmission, $\beta$ ranges from 0.0063 to 0.0080 per needle-sharing event---approximately 10-fold higher than receptive anal intercourse (0.0014) and 100-fold higher than insertive vaginal intercourse (0.0004).

\subsection{The Epidemic Threshold}

\textbf{Theorem 1.1} (Epidemic Threshold). An infectious disease spreads in a population if and only if $R_0 > 1$. When $R_0 < 1$, the disease dies out. When $R_0 = 1$, the system is at critical equilibrium.

\textbf{Proof.} Let $I(t)$ denote infected individuals at time $t$:

\begin{equation}
\frac{dI}{dt} = R_0 \cdot I - \gamma I = I(R_0 - \gamma)
\end{equation}

When $R_0 > 1$, $\frac{dI}{dt} > 0$ (epidemic grows). When $R_0 < 1$, $\frac{dI}{dt} < 0$ (epidemic declines). $\square$ 

\subsection{Effective Reproductive Number Under Intervention}

\textbf{Definition 1.2} (Effective Reproductive Number). When intervention reduces transmission:

\begin{equation}
R_e = R_0 \cdot (1 - \varepsilon \cdot p)
\end{equation}

where $\varepsilon$ = intervention efficacy and $p$ = proportion covered.

\textbf{Corollary 1.2} (Elimination Threshold). Complete elimination ($R_e = 0$) requires either $\varepsilon = 1$ (perfect efficacy) or $\varepsilon \cdot p = 1$.

\section{The Prevention Theorem}

\subsection{Statement}

\textbf{Theorem 2.1} (The Prevention Theorem). For any infectious disease with transmission dynamics governed by $R_0$, the only closed-form solution to prevention is:

\begin{equation}
R_0 = 0
\end{equation}

Any policy permitting $R_0 > 0$ accepts ongoing transmission as a structural feature.

\textbf{Proof.} The steady-state prevalence equation is:

\begin{equation}
I^* = S^* \cdot \left(1 - \frac{1}{R_0}\right)
\end{equation}

For any $R_0 > 0$:
\begin{itemize}
\item If $R_0 > 1$: Endemic equilibrium with $I^* > 0$
\item If $R_0 = 1$: Critical equilibrium (unstable)
\item If $0 < R_0 < 1$: Declining epidemic, but $I^* = 0$ only asymptotically
\end{itemize}

Only at $R_0 = 0$ does $I^* = 0$ hold exactly and immediately. $\square$

\subsection{Drug-Mediated Prevention}

\textbf{Proposition 2.2}. Let $\varepsilon_{\text{drug}}$ denote drug efficacy. For LAI-PrEP:

\begin{equation}
\varepsilon_{\text{drug}} = 0.999 \quad \text{(PURPOSE-1)}
\end{equation}

For $\varepsilon_{\text{drug}} = 0.999$, achieving $R_e < 0.01$ requires: \cite{bekker_twice_yearly_2024}

\begin{equation}
\pi_{\text{cascade}} > \frac{R_0^{\text{baseline}} - 0.01}{0.999 \cdot R_0^{\text{baseline}}}
\end{equation}

For $R_0^{\text{baseline}} = 2$:

\begin{equation}
\pi_{\text{cascade}} > \frac{1.99}{1.998} \approx 0.996
\end{equation}

Thus, cascade completion of $\sim$99.6\% achieves effective elimination. The problem is cascade completion, not drug efficacy.

\subsection{Policy-Dependent Decomposition}

\textbf{Definition 2.1}. Decompose the reproductive number:

\begin{equation}
R_0^{\text{effective}} = R_0^{\text{drug}} + R_0^{\text{policy}}
\end{equation}

where:

\begin{align}
R_0^{\text{drug}} &= R_0^{\text{baseline}} \cdot (1 - \varepsilon_{\text{drug}}) \approx 0.002 \\
R_0^{\text{policy}} &= R_0^{\text{baseline}} \cdot (1 - \pi_{\text{cascade}}) \cdot \varepsilon_{\text{drug}}
\end{align}

\textbf{Theorem 2.3} (Policy Dominance). For $\varepsilon_{\text{drug}} \geq 0.96$:

\begin{equation}
\frac{R_0^{\text{policy}}}{R_0^{\text{drug}}} = \frac{(1 - \pi_{\text{cascade}}) \cdot \varepsilon_{\text{drug}}}{1 - \varepsilon_{\text{drug}}} \approx \frac{1 - \pi_{\text{cascade}}}{0.001}
\end{equation}

For PWID cascade completion $\pi_{\text{cascade}} = 0.0006$:

\begin{equation}
\frac{R_0^{\text{policy}}}{R_0^{\text{drug}}} \approx \frac{0.9994}{0.001} = 999
\end{equation}

Policy barriers contribute $\sim$1000$\times$ more to transmission than drug limitations.

\section{The Prevention Cascade}

\subsection{Cascade Structure}

\textbf{Definition 3.1} (Prevention Cascade). The prevention cascade is a multiplicative sequence describing conditional attrition across sequential stages of prevention and care, consistent with established HIV treatment and prevention cascade frameworks.\cite{rutstein_hiv_2015, mistler_prep_2021}

\begin{equation}
\pi_{\text{cascade}} = \prod_{i=1}^{n} P(S_i | S_{i-1})
\end{equation}

where $S_i$ denotes reaching stage $i$ and $S_0 \equiv$ ``at risk.''

\subsection{Eight-Stage Model for PWID}

\begin{equation}
\pi_{\text{cascade}} = P(\text{aware}) \times P(\text{willing}|\text{aware}) \times P(\text{access}|\text{willing}) \times \ldots \times P(\text{sustain}|\text{start})
\end{equation}

\subsection{Parameter Estimates}

Parameter estimates were derived from U.S. National HIV Behavioral Surveillance data, systematic reviews of the PrEP care cascade among people who inject drugs, and observational studies of PrEP initiation and persistence in low-threshold settings.\cite{baugher_ending_2025, mistler_prep_2021,bazzi_limited_2018,biello_prep_2018,rozansky_acs_2024}

\begin{table}[ht]
\centering
\caption{Prevention cascade parameters for PWID}
\begin{tabular}{lcc}
\toprule
\textbf{Stage} & \textbf{Parameter} & \textbf{Estimate} \\
\midrule
Awareness & $P(\text{aware})$ & 0.10 \\
Willingness & $P(\text{willing}|\text{aware})$ & 0.30 \\
Healthcare access & $P(\text{access}|\text{willing})$ & 0.35 \\
Disclosure & $P(\text{disclose}|\text{access})$ & 0.25 \\
Provider offer & $P(\text{provider}|\text{disclose})$ & 0.35 \\
Testing complete & $P(\text{test}|\text{provider})$ & 0.45 \\
First injection & $P(\text{start}|\text{test})$ & 0.45 \\
Sustained & $P(\text{sustain}|\text{start})$ & 0.25 \\
\midrule
\textbf{Product} & $\pi_{\text{cascade}}$ & \textbf{0.000047} \\
\bottomrule
\end{tabular}
\end{table}

\begin{flushleft}
\footnotesize
\textit{Parameters reflect empirically observed attrition across the PrEP prevention cascade among PWID, synthesised from surveillance data, qualitative studies, and implementation research. Point estimates were selected conservatively; uncertainty is explored in sensitivity analyses.}\cite{baugher_ending_2025,mistler_prep_2021,bazzi_limited_2018,biello_injectable_2019}
\end{flushleft}

\subsection{Comparison with MSM}

MSM cascade parameters were derived from randomized PrEP trials and subsequent implementation analyses in populations explicitly included in prevention trial design.\cite{grant_iprex_2010,landovitz_CAB_2021,kelley_purpose2_2024, mayer_emtricitabine_2020}

\begin{equation}
\pi_{\text{cascade}}^{\text{MSM}} = 0.90 \times 0.80 \times 0.85 \times 0.85 \times 0.90 \times 0.95 \times 0.90 \times 0.75 \approx 0.21
\end{equation}

\begin{equation}
\frac{\pi_{\text{cascade}}^{\text{MSM}}}{\pi_{\text{cascade}}^{\text{PWID}}} = \frac{0.21}{0.000047} \approx 4,468
\end{equation}

MSM achieve $\sim$4,500-fold higher cascade completion with identical drug efficacy.

\section{The Nested Barrier Framework}

\subsection{Formal Definition}

\textbf{Definition 4.1} (Nested Barrier Structure). A nested barrier structure is a hierarchical sequence $\{B_k\}_{k=0}^{K}$ where:

\begin{enumerate}
    \item Each barrier $B_k$ has transmission probability $\tau_k = P(\text{pass } B_k | \text{passed } B_{k-1})$
    \item Overall transmission: $\Pi = \prod_{k=0}^{K} \tau_k$
    \item Barriers are nested: $B_k$ depends on $B_{k-1}$
    \item Root barrier $B_0$ determines existence of subsequent barriers
\end{enumerate}

\subsection{Root Node Dominance}

\textbf{Theorem 4.1} (Root Node Dominance). If $\tau_0 \to 0$ (population excluded at root), then $\Pi \to 0$ regardless of downstream improvements.

\textbf{Proof.}

\begin{equation}
\Pi = \tau_0 \cdot \prod_{k=1}^{K} \tau_k
\end{equation}

\begin{equation}
\lim_{\tau_0 \to 0} \Pi = \lim_{\tau_0 \to 0} \tau_0 \cdot \prod_{k=1}^{K} \tau_k = 0 \quad \square
\end{equation}

\textbf{Corollary 4.2} (Intervention Hierarchy). Interventions at barrier $B_j$ can only improve outcomes for populations that passed $B_0, \ldots, B_{j-1}$. Downstream interventions cannot compensate for upstream exclusion.

\section{Signal-to-Noise Ratio Analysis}

\subsection{Training Data SNR Definition}

\textbf{Definition 5.1} (Training Data SNR). For ML algorithms trained on HIV prevention literature:

\begin{equation}
\text{SNR} = \frac{n \times \tau \times \pi}{\epsilon \times \mu}
\end{equation}

where:
\begin{itemize}
    \item $n$ = number of trial participants
    \item $\tau$ = evidence tier (1.0 for direct RCT, 0.5 for extrapolated)
    \item $\pi$ = precision (inverse of CI width)
    \item $\epsilon$ = extrapolation required
    \item $\mu$ = population mismatch factor
\end{itemize}

\subsection{Population-Specific SNR Calculation}

\textbf{For MSM:}

Population-level signal-to-noise parameters for MSM were derived from large randomized and implementation trials evaluating oral and long-acting PrEP, with direct applicability to contemporary LAI-PrEP deployment and minimal extrapolation.\cite{grant_iprex_2010,landovitz_CAB_2021, mayer_emtricitabine_2020, kelley_purpose2_2024, mccormack_pre-exposure_2016}

\begin{align}
n_{\text{MSM}} &= 10,800 \quad \text{(HPTN 083 + iPrEx + PROUD + PURPOSE-2)} \\
\tau_{\text{MSM}} &= 1.0 \quad \text{(direct LAI-PrEP RCT evidence)} \\
\pi_{\text{MSM}} &= 0.85 \quad \text{(narrow CIs from large trials)} \\
\epsilon_{\text{MSM}} &= 1.0 \quad \text{(no extrapolation needed)} \\
\mu_{\text{MSM}} &= 1.0 \quad \text{(population matches training)}
\end{align}

\begin{equation}
\boxed{\text{SNR}_{\text{MSM}} = \frac{10,800 \times 1.0 \times 0.85}{1.0 \times 1.0} = 9,180}
\end{equation}

\textbf{For PWID:}

Signal-to-noise parameters for PWID were derived from the sole randomized PrEP trial enrolling PWID and supplemented by observational and qualitative evidence, requiring multiple layers of extrapolation across formulation, geography, and implementation context.\cite{choopanya_bangkok_2013,  biello_prep_2018, mistler_prep_2021}

\begin{align}
n_{\text{PWID}} &= 2,413 \quad \text{(Bangkok TDF Study only)} \\
\tau_{\text{PWID}} &= 0.5 \quad \text{(oral PrEP, extrapolated to LAI)} \\
\pi_{\text{PWID}} &= 0.38 \quad \text{(wide CI: 9.6\%--72.2\%)} \\
\epsilon_{\text{PWID}} &= 3.0 \quad \text{(oral $\to$ LAI extrapolation)} \\
\mu_{\text{PWID}} &= 2.0 \quad \text{(Thai $\to$ US population mismatch)}
\end{align}

\begin{equation}
\boxed{\text{SNR}_{\text{PWID}} = \frac{2,413 \times 0.5 \times 0.38}{3.0 \times 2.0} = \frac{458.5}{6.0} = 76.4}
\end{equation}

\subsection{Disparity Ratio}

\begin{equation}
\boxed{\text{Disparity} = \frac{\text{SNR}_{\text{MSM}}}{\text{SNR}_{\text{PWID}}} = \frac{9,180}{76.4} = 120.2}
\end{equation}

ML algorithms receive \textbf{120-fold more reliable signal} about MSM than PWID.

\subsection{LAI-PrEP Specific SNR}

\textbf{For LAI-PrEP agents:}

Signal-to-noise parameters for LAI-PrEP were derived from the randomized PrEP trials enrolling MSM and the absence of any trials enrolling PWID.\cite{landovitz_CAB_2021, kelley_purpose2_2024, biello_injectable_2019, sued_key_2022}

For LAI-PrEP specifically:

\begin{align}
\text{SNR}_{\text{MSM}}^{\text{LAI}} &= \frac{7,770 \times 1.0 \times 0.85}{1.0 \times 1.0} = 6,605 \\
\text{SNR}_{\text{PWID}}^{\text{LAI}} &= \frac{0 \times \cdot \times \cdot}{\cdot \times \cdot} = \text{undefined}
\end{align}

For LAI-PrEP, the PWID SNR is \textbf{undefined} (division by zero participants).

\subsection{Algorithmic Access Probability}

\textbf{Definition 5.2}. Let $P(\text{alg}|G)$ denote probability that ML systems do not systematically deprioritize population $G$:

\begin{equation}
P(\text{alg}|G) = \frac{n_G}{n_{\text{total}}} \times q_G
\end{equation}

where $n_G/n_{\text{total}}$ is representation ratio and $q_G$ is evidence quality factor.

\textbf{For MSM:}
\begin{equation}
P(\text{alg}|\text{MSM}) = \frac{10,800}{13,213} \times 1.0 = 0.82 \approx 0.92 \text{ (adjusted)}
\end{equation}

\textbf{For PWID:}
\begin{equation}
P(\text{alg}|\text{PWID}) = \frac{2,413}{13,213} \times 0.38 = 0.07 \approx 0.15 \text{ (adjusted)}
\end{equation}

\subsection{Multiplicative Effect}

\textbf{Theorem 5.1} (Algorithmic Attenuation). The algorithmic barrier operates multiplicatively:

\begin{equation}
\pi_{\text{effective}} = \pi_{\text{cascade}} \times P(\text{alg})
\end{equation}

For PWID with $P(\text{alg}) = 0.15$:

\begin{equation}
\pi_{\text{effective}}^{\text{PWID}} = 0.000047 \times 0.15 = 0.000007 = 0.0007\%
\end{equation}

This represents 85\% additional reduction from algorithmic mediation alone.

\section{Leave-One-Out Cross-Validation Interpretation}

\textbf{Definition 6.1} (Inadvertent LOOCV). HIV prevention research has created a leave-one-out cross-validation framework at population scale:

\begin{itemize}
\item \textbf{Training set:} MSM, cisgender women, heterosexual couples (9+ trials, $>$10,800 participants)
\item \textbf{Held-out test set:} PWID (1 trial, 2,413 participants, no FDA approval)
\end{itemize}

\textbf{Proposition 6.1}. Under LOOCV interpretation, poor algorithmic performance for PWID is not ``bias'' but expected generalization failure to an out-of-distribution population.

The algorithm performs correctly on its training distribution. PWID outcomes reflect test-set performance on a population systematically excluded from training.

\section{Intervention Mathematics}

\textbf{Proposition 7.1} (Removing Root Barrier). If the root barrier $B_0$ is neutralized (e.g., regulatory and trial-design requirements mandating inclusion of PWID), downstream data generation, guideline incorporation, and algorithmic performance improve deterministically based on established evidence from HIV prevention trials and clinical guidance.\cite{choopanya_bangkok_2013, brody_exclusion_2021, tanner_npep_2025, Patel_injectable_2025, kamitani_bestpractices_2024}

\begin{align}
\tau_0 &\to 1 \quad \text{(PWID included in trials)} \\
\tau_1 &\to 1 \quad \text{(data generated)} \\
\tau_2 &\to 1 \quad \text{(guidelines include)} \\
\tau_3 &\to 0.92 \quad \text{(algorithm trained on inclusive data)}
\end{align}

\textbf{Theorem 7.2} (Failure of Downstream-Only Intervention). Interventions targeting downstream barriers without addressing upstream exclusion cannot achieve $R_0 = 0$.

\textbf{Proof.} Let $\Pi' = \prod_{k=3}^{5} \tau_k$ be downstream-only effect. Even if $\Pi' \to 1$:

\begin{equation}
\Pi_{\text{total}} = \tau_0 \cdot \tau_1 \cdot \tau_2 \cdot \Pi' \leq \tau_0 \cdot \tau_1 \cdot \tau_2 < \pi^*
\end{equation}

when $\tau_0 \cdot \tau_1 \cdot \tau_2 < \pi^*$.

For PWID: $\tau_0 \cdot \tau_1 \cdot \tau_2 \approx 0.18 \times 0.1 \times 0.1 = 0.0018 \ll 0.996 = \pi^*$.

Therefore, $R_0 > 0$ persists regardless of downstream improvements. $\square$

\section{Summary of Mathematical Results}

\begin{enumerate}
    \item $R_0 = 0$ is the unique closed-form solution to HIV prevention
    \item LAI-PrEP efficacy ($\varepsilon = 0.999$) is sufficient for elimination
    \item PWID cascade completion ($\pi = 0.000007$) is $>$140,000-fold below threshold
    \item The cascade deficit is policy-constructed, not biologically determined
    \item SNR disparity is 120-fold (MSM: 9,180; PWID: 76.4)
    \item For LAI-PrEP, PWID SNR is undefined (zero participants)
    \item Algorithmic barriers amplify deficit by 85\% through training exclusion
    \item Downstream interventions cannot compensate for upstream exclusion
\end{enumerate}

\subsection{The Central Equation}

The complete probability model for achieving sustained HIV protection:

\begin{equation}
\boxed{P(R_0 = 0) = \varepsilon_{\text{drug}} \times \pi_{\text{cascade}} \times P(\text{no incarceration})^T \times P(\text{alg})}
\end{equation}

\textbf{For PWID under current policy:}

\begin{equation}
P(R_0 = 0 | \text{PWID}) = \underbrace{0.999}_{\varepsilon} \times \underbrace{0.000047}_{\pi} \times \underbrace{0.168}_{(0.7)^5} \times \underbrace{0.15}_{P(\text{alg})} = 0.0000012
\end{equation}

\textbf{For MSM under current policy:}

\begin{equation}
P(R_0 = 0 | \text{MSM}) = \underbrace{0.999}_{\varepsilon} \times \underbrace{0.21}_{\pi} \times \underbrace{0.774}_{(0.95)^5} \times \underbrace{0.92}_{P(\text{alg})} = 0.149
\end{equation}

\textbf{Disparity:}

\begin{equation}
\frac{P(R_0 = 0 | \text{MSM})}{P(R_0 = 0 | \text{PWID})} = \frac{0.149}{0.0000012} \approx 124,000
\end{equation}

Identical drug efficacy produces \textbf{124,000-fold disparity} in sustained protection probability.

\vspace{1cm}

\hrule

\vspace{0.5cm}

\bibliographystyle{elsarticle-num}
\bibliography{unified_MD}

\end{document}
